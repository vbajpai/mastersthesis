%$F(v1)$ has been a great headstart to provide fast execution of stream-based
%flow records. However the execution engine only hits one of the several stages
%of the processing pipeline.

This section dives deep into the implementation of the next iteration $F(v2)$
of the \ac{NFQL} execution engine. This iteration provides a functional and
robust implementation of the complete processing pipeline. It is flexible to
allow runtime evaluation of \ac{NFQL} flowqueries and is backed up with
automated builds, regression and benchmarking suites. The organization of the
section is described below.

\vspace{50pt}

In chapter $\ref{ch:design}$ we begin by analyzing the current implementation
snapshot. The analysis involves reverse-engineering the parser and the
execution engine. This is followed by a discussion on an early visualization
of a complete engine refactor using abstract objects and reasonings on an
envisaged high-level execution workflow.

In chapter $\ref{ch:implementation}$ we introduce the inner workings of the
code. It begins by an explanation of how each stage is brought to life to
result in a robust pipeline exeuction. The grouper and merger internals, being
non-trivial are explained in more details. This is followed by an explanation
on how the engine is engineered to make runtime query evaluation a
possibility. The chapter concludes with a discussion on automated builds using
\texttt{cmake} and a full-fledged regression-test suite using \texttt{python}
scripts.

In chapter $\ref{ch:performance-evaluation}$ we begin by a discussion of the
execution engine profiling results. This is followed by how \texttt{python}
scripts are used to completely automate the process of benchmarking the engine
against SiLK. A number of queries are run a collection of varied-sized flow
traces. The chapter concludes with a discussion of the evaluation graphs. We
conclude the discussion in chapter $\ref{ch:future-work}$ by documenting the
future work by segregating it into major goals and minor issues that need to
be resolved to carry the implementation work forward.
