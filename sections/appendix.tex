The appendix is used to supplement the work done in the thesis with clear
instructions on how to use the end-product. It also includes instructions on
how to build and use other products that were used as reference points when
conducting performance evaluation to allow the process to be easily repeatable
by others. The organization of the appendix is described below.

\vspace{50pt}

In section $\ref{ch:nfql-installation-usage}$ we discuss step-by-step
instructions on how to install \ac{NFQL} parser and the execution engine on
Debian-based systems and OS X. This walkthrough is going to help not only the
users, but also future developers to quickly get started with \ac{NFQL}
implementations.

In section $\ref{ch:silk-installation-usage}$ we discuss SiLK installation on
Debian-based systems. We then investigate which SiLK tool can be used to
achieve the desired effect of each stage of the \ac{NFQL} processing pipeline.
Additional SiLK analysis and capture tools are also discussed. We end the
section, by enumerating the process of converting \texttt{flow-tools}
compatible traces to SiLK proprietory format.

In section $\ref{ch:release-notes}$ we enlist the major iterations of the
development lifecycle of the execution engine starting from $F(v2.0)$ to
$F(v2.5)$. An explanation of the feature set in each iteration is followed by
instructions on how \texttt{git tags} can be used to go back to a previous
version. The appendix is concluded by a list of acronyms used in this work.
