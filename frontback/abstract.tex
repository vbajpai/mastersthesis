%*******************************************************
% Abstract
%*******************************************************
%\renewcommand{\abstractname}{Abstract}
\pdfbookmark[1]{Abstract}{Abstract}
\begingroup
\let\clearpage\relax
\let\cleardoublepage\relax
\let\cleardoublepage\relax

\chapter*{Abstract}

With the long dominance of Cisco's NetFlow \cite{rfc3954} protocol and now with the emergence of \ac{IETF}'s \ac{IPFIX} \cite{rfc5101} open standard, traffic measurement practitioners have finally settled down with using \ac{IP} flow export as the de-facto technique for sending traffic patterns. These patterns have the potential to be used for billing and mediation, bandwidth provisioning, detecting malicious attacks, network performance evaluation and overall improvement. \\

However, making sense of these patterns calls for sophisticated flow-analysis tools that can mine them for such a usage. Unfortunately current tools fail to deliver owing to their poor language design and n\"aive filtering methods. Our research group, by going clean slate has come up with a flow-query language design \cite{vmarinov:thesis:2009} that can cap the flow-exports to full potential. The flow-query language can process flow-records, aggregate them into groups, apply absolute (or relative) filters and invoke Allen interval algebra rules \cite{fallen:1983} on these records. \\

F \cite{jschauer:2012} is the prototype implementation of our in-house flow query language which has underwent significant changes in the last few years. The core of the former Python implementation \cite{kkanev:thesis:2009} has now been rewritten in C \cite{jschauer:thesis:2011} to make it comparable to the contemporary flow processing tools. However, this has disconnected the flow-query parser present in the former implementation. The two implementations have now branched off so much that both currently live in their own parallel universe. This thesis takes up the challenge to glue the better parts of both of these implementations together to create a complete package that has the full-blown functionality and exploits the best of both worlds. In the process, it is also planned to bring the implementation up to speed with bleeding edge \ac{IPFIX} support, parallelize it by making it MapReduce \cite{jdean:2004} aware and recover it from limitations learnt from the wealth of experience gained after managing the two branches for the last few years. 



\endgroup			
\vfill