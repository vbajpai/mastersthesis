SiLK uses its own proprietary format for reading flow traces. In order to be
able to perform the evaluation on the same flow traces, it was essential to
convert the \texttt{flow-tools}
\footnote{\url{http://www.splintered.net/sw/flow-tools/}} format trace files
to SiLK proprietary format. The best way is to replay the original trace files
on a \marginpar{flow-tools to nfdump} host-port combination, and let SiLK's
flow capturing daemon pick it up to save it in its proprietory format.
Unfortunately, \texttt{flow-tools} does not have any such replay tool.
Although, one is provided by the \texttt{nfdump}
\footnote{\url{http://nfdump.sourceforge.net/}} package. Listing
\ref{lst:ft2nfdump} shows how \texttt{flow-tools} traces were converted to
\texttt{nfdump} format to allow them to be replayed later.

\lstset{caption=\texttt{flow-tools} to \texttt{nfdump},
				tabsize=2, language=bash, numbers=left,stepnumber=1,
        basicstyle=\tiny\ttfamily, numberstyle=\ttfamily\color{gray},
        keywordstyle=\color{blue}, frame=shadowbox,
        rulesepcolor=\color{black}, label=lst:ft2nfdump,
        aboveskip=20pt, captionpos=b, upquote=true}
\begin{lstlisting}
# install nfdump and ft2nfdump
$ sudo apt-get install nfdump
$ sudo apt-get install nfdump-flow-tools

# convert flow-tools traces to nfdump
$ flow-cat $INPUT | ft2nfdump | nfdump -w $OUTPUT
$ nfdump -r $OUTPUT
\end{lstlisting}

The converted \texttt{nfdump} traces were then replayed using
\texttt{nfreplay}. By default, the tools replays the traces to $127.0.0.1$ at
port $9995$. A sensor configuration file was created on the other end for SiLK
to collect the replayed data. The configuration file defines the NetFlow
protocol used in the replay, host-port combination and definition of internal
and external \ac{IP} blocks to separate \marginpar{nfdump to silk} the
flow-traces accordingly. SiLK's \texttt{rwflowpack} was then used to regenerate
the traces in the proprietary format. The tool segregates the flows into
multiple hierarchically arranged directories. Since, only the runtime analysis
of SilK on the original query was of interest, a combination of \texttt{find}
and \texttt{xargs} was used to flattern the directory and combine the flows
into one file as shown in listing \ref{lst:nfdump2silk}.

\lstset{caption=\texttt{nfdump} to SiLK,
				tabsize=2, language=bash, numbers=left,stepnumber=1,
        basicstyle=\tiny\ttfamily, numberstyle=\ttfamily\color{gray},
        keywordstyle=\color{blue}, frame=shadowbox,
        rulesepcolor=\color{black}, label=lst:nfdump2silk,
        aboveskip=20pt, captionpos=b, upquote=true}
\begin{lstlisting}
# replay the nfdump trace
$ nfreplay -r $TRACE

# configure a sensor to collect replayed data
$ cat sensors.conf
probe S1 netflow-v5
  listen-on-port 9995
  protocol udp
  accept-from-host 127.0.0.1
end probe
sensor S1
  netflow-v5-probes S1
  internal-ipblocks 10.0.0.0/8
  external-ipblocks remainder
end sensor

# collect flow data and save in binary silk files
$ rwflowpack \
  --site-config-file=/usr/local/share/silk/generic-silk.conf \
  --sensor-configuration=sensors.conf \
  --root-directory=/var/log/silk/ \
  --log-destination=both

rwflowpack[14830]: Forked child 14831.  Parent exiting
rwflowpack[14831]: Using packing logic from ...
rwflowpack[14831]: Creating stream cache
rwflowpack[14831]: Starting flow processor #1 for PDU Reader
rwflowpack[14831]: Creating PDU Reader Source Pool
rwflowpack[14831]: Creating PDU Reader for probe 'S1' on 0.0.0.0:9995
rwflowpack[14831]: Starting flush timer
rwflowpack[14831]: Started manager thread for PDU Reader

# flatten the silk root directory
$ find /var/log/silk -type f -exec cp {} /var/log/silkflat/ \;

# combine all silk files into a single archive
$ ls | rwcat --xargs --output-path=/var/log/silk.gz
\end{lstlisting}
