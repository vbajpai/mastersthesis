The execution engine is now portable to be able to seamlessly build on
multiple Unix flavors, tagged as $v0.4$. CMake is used to orchestrate the buid
process. It uses custom commands to invoke scripts that can prepare the
auto-generated \marginpar{it's portable} sources and sample \texttt{JSON}
queries. A Makefile is used to automate the CMake invocations. Feature-test
and platform-specific macros allow the code to become portable. The snapshot
contains installation instructions for both Debian/Ubuntu and OS X. A workset
of the release is shown in listing \ref{lst:v0.4}.

\lstset{caption=Release Notes: $v0.4$,
				tabsize=2, language=bash, numbers=left,stepnumber=1,
        basicstyle=\tiny\ttfamily, numberstyle=\ttfamily\color{gray},
        keywordstyle=\color{blue}, frame=shadowbox,
        rulesepcolor=\color{black}, label=lst:v0.4,
        aboveskip=20pt, captionpos=b, upquote=true}
\begin{lstlisting}
$ git show v0.4

tag v0.4
Tagger: Vaibhav Bajpai <contact@vaibhavbajpai.com>
Date:   Fri May 18 15:07:42 2012 +0200
Commit 00c17385e37dd944c9139205a5eb3660c707858a

*  _GNU_SOURCE feature test MACRO and -std=c99
*  (__FreeBSD, __APPLE__) and __linux MACROS around qsort_r(...)
*  reverted to a flat source structure for the CMake build process.
*  CMake custom command to call a script to create auto-generated sources and headers.
*  CMake custom command to call a scripts in queries/ to save sample JSON queries in examples/
*  Makefile to automate invocation of CMake commands.
*  installation instruction for Ubuntu.
*  installation instruction for Mac OS X.
\end{lstlisting}

