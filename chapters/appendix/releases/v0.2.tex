The execution engine is robust to work well with variety of queries, tagged as
$v0.2$. The complete engine has been refactored with a maintainable design to
allow better execution workflow and abstract objects. Each stage of the
pipeline has one public interface function that takes a ruleset as input and
returns a result abstract object. The engine has \marginpar{it's robust} been
profiled to eliminate any memory leaks and allow early deallocation of objects
as soon as they are not required. The engine is now smart enough to realize
and ignore redundant aggregation requests. All the hardcoded rules of the
flowquery are clubbed together in one header file for easy maintainability.
The dedicated rule function pointer assignments are lazy. A workset of the
release state is shown in listing \ref{lst:v0.2}.

\lstset{caption=Release Notes: $v0.2$,
				tabsize=2, language=bash, numbers=left,stepnumber=1,
        basicstyle=\tiny\ttfamily, numberstyle=\ttfamily\color{gray},
        keywordstyle=\color{blue}, frame=shadowbox,
        rulesepcolor=\color{black}, label=lst:v0.2,
        aboveskip=20pt, captionpos=b, upquote=true}
\begin{lstlisting}
$ git show v0.2

tag v0.2
Tagger: Vaibhav Bajpai <contact@vaibhavbajpai.com>
Date:   Wed Apr 18 13:24:16 2012 +0200
Commit 2c571f80cd076172cbd00ef7f9976b88cb44b425

* complete engine refactor.
* complete engine profiling (no memory leaks).
* issues closed:
    - greedily deallocating non-filtered records in O(n) before merger(...).
    - resolved a grouper segfault when NO records got filtered.
    - all records are grouped into 1 group when no grouping rule specified.
    - aggregation on common fields touched by filter/grouper rules is ignored.
    - no uintX_t assumptions for field offsets.
    - rules are clubbed together and assigned using a loop.
    - function parameters are as minimum as required.
    - function parameters are safe using [const] ptr and ptr to [const].
    - lazy rule->func(...) assignment when the stage is entered.
\end{lstlisting}

