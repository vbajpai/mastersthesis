This tag starts off the F$(v2)$ branch.  The complete pipeline now works for
the first time, tagged as $v0.1$. The grouper segfaults have been resolved.
The resulting group records are cooked as NetFlow $v5$ records with their
field offsets representing group aggregations. \marginpar{it's functional}
This snapshot contains the first-ever group filter, merger and ungrouper
implementation in C. The stages do not assume type of the field offsets that
are not known until \emph{runtime}. The engine now works with multiple
verbosity levels increasing the amount of echo on each level. A workset of
the release is shown in listing \ref{lst:v0.1}.

\lstset{caption=Release Notes: $v0.1$,
				tabsize=2, language=bash, numbers=left,stepnumber=1,
        basicstyle=\tiny\ttfamily, numberstyle=\ttfamily\color{gray},
        keywordstyle=\color{blue}, frame=shadowbox,
        rulesepcolor=\color{black}, label=lst:v0.1,
        aboveskip=20pt, captionpos=b, upquote=true}
\begin{lstlisting}
$ git show v0.1

tag v0.1
Tagger: Vaibhav Bajpai <contact@vaibhavbajpai.com>
Date:   Fri Apr 6 19:07:49 2012 +0200
Commit a8a67a13aa07f671d21d062537a2ef17e58dcc07
...

* reverse engineered parser to generate UML.
* froze requirements to allow single step installation of the python parser.
* doxygen documentation of the engine.
* prelim JSON parsing framework for the parser and engine to spit and parse the JSON queries.
* replaced GNU99 extensions dependent code with c99.
* resolved numerous segfaults in the grouper.
* generated group aggregations as a separate (cooked) NetFlow v5 record.
* flexible group aggregations with no uintX_t assumptions on field offsets.
* first-ever group filter implementation.
* reorganized the src/ directory structure
* enabled multiple verbosity levels in the engine.
* first-ever merger implementation.
* flexible filters and group filters with no uintX_t assumptions on field offsets.
* first-ever ungrouper implementation.
\end{lstlisting}

