The \ac{NFQL} parser uses the \texttt{pip} packaging and installation
environment to set itself up. Besides some of the external dependencies
required by the parser, it is highly recommended to use \texttt{virtualenv}
and \texttt{virtualenvwrapper} \marginpar{nfql parser on debian/ubuntu} to
create a virtual environment where all the python libraries will be installed.
This helps isolate the user's system-level python libraries and avoids any
conflicts that may occur otherwise. The external dependencies and \texttt{pip}
environment installation is shown in listing
\ref{lst:nfql-parser-ubuntu-dependencies}.

\lstset{caption=NFQL Parser Dependencies on Debian/Ubuntu,
				tabsize=2, language=bash, numbers=left,stepnumber=1,
        basicstyle=\tiny\ttfamily, numberstyle=\ttfamily\color{gray},
        keywordstyle=\color{blue}, frame=shadowbox,
        rulesepcolor=\color{black}, label=lst:nfql-parser-ubuntu-dependencies,
        aboveskip=20pt, captionpos=b, upquote=true}
\begin{lstlisting}
$ sudo apt-get install libhdf5-serial-dev
$ sudo apt-get install liblzo2-dev

$ sudo apt-get install python-pip
$ sudo pip install pip --upgrade
$ sudo pip install virtualenv
$ sudo pip install virtualenvwrapper
\end{lstlisting}

A python virtual environment is setup from the parent directory of the parser
by running \texttt{mkvirtualenv}. Once within the virtual environment,
\texttt{make} takes care of installing all the python libraries required by
the parser in one go. It installs \texttt{cython}, \texttt{numpy},
\texttt{numexpr} in a preprocessing step, and then lets \texttt{pip} handle
the rest of the installation using a \texttt{requirements} file. The list of
python libraries installed can be checked by running \texttt{pip freeze}. The
parser usage instructions are provided in the next section. The virtual
environment can be deactivated using \texttt{deactivate} and brought back
again using \texttt{workon \$NAME}. It can be destroyed using
\texttt{rmvirtualenv} as shown in listing \ref{lst:nfql-parser-build}

\lstset{caption=Platform-Agnostic NFQL Parser Build,
				tabsize=2, language=bash, numbers=left,stepnumber=1,
        basicstyle=\tiny\ttfamily, numberstyle=\ttfamily\color{gray},
        keywordstyle=\color{blue}, frame=shadowbox,
        rulesepcolor=\color{black}, label=lst:nfql-parser-build,
        aboveskip=20pt, captionpos=b, upquote=true}
\begin{lstlisting}
[parser] $ mkvirtualenv parser
(parser)
[parser] $ make
(parser)
[parser] $ pip freeze

(parser)
[parser] $ python ft2hdf.py traces/ output.h5
(parser)
[parser] $ python printhdf.py output.h5
(parser)
[parser] $ python print_hdf_in_step.py output.h5
(parser)
[parser] $ python flowy.py queryfile

(parser)
[parser] $ make clean
(parser)
[parser] $ deactivate

[parser] $ rmvirtualenv parser
\end{lstlisting}
