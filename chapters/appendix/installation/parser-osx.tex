\lstset{caption=NFQL Parser Dependencies on OS X,
				tabsize=2, language=bash, numbers=left,stepnumber=1,
        basicstyle=\tiny\ttfamily, numberstyle=\ttfamily\color{gray},
        keywordstyle=\color{blue}, frame=shadowbox,
        rulesepcolor=\color{black}, label=lst:nfql-parser-osx-dependencies,
        aboveskip=20pt, captionpos=b, upquote=true}
\begin{lstlisting}
$ brew install hdf5
$ brew install lzo

$ brew install python --framework
$ export PATH=/usr/local/share/python:$PATH

$ easy_install pip
$ pip install pip --upgrade
$ pip install virtualenv
$ pip install virtualenvwrapper
$ source /usr/local/bin/virtualenvwrapper.sh
\end{lstlisting}

The \ac{NFQL} parser installation on OS X only differs in the \emph{way} how
external libraries are installed. Homebrew is used to install the external
packages.  The default python framework on OS X is fairly old, it is in the
best interest to also upgrade it using Homebrew. The older python packaging
\marginpar{nfql parser on osx} environment \texttt{easy\_install} is used to
install \texttt{pip}, and then \texttt{pip} is used to upgrade itself to the
latest revision. The packaging from there is on handled by the \texttt{pip}
environment. The external dependencies and \texttt{pip} environment
installation is shown in listing \ref{lst:nfql-parser-osx-dependencies}, while
the virtual environment setup and parser build process is exactly the same as
shown in listing \ref{lst:nfql-parser-build}.
