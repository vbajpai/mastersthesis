%**************************************
\chapter{Flowy $2.0$}\label{ch:flowy-2}
%**************************************

In an attempt to make the first prototype implementation of Flowy comparable
with the contemporary flow-record analysis tools, the substitution of the
performance hit sections of the Python code was thought out. Flowy $2.0$
\cite{jschauer:thesis:2011} is the outcome of a complete rewrite of the core
of the prototype implementation in C making it relatively faster in orders of
magnitude.

\section{Performance Issues}\label{sec:performance-issues}
\begin{table}[h!]
	\begin{tabular}{|c|c|c|c|c|}
	\hline
	no. of records & overall & filter & grouper & merger \\
	\hline
	\hline
	$103k$ & $1177s$ & $28s (2\%)$ & $240s (20\%)$ & $909s (77\%)$\\
	\hline
	$337k$ & $20875s$ & $110s (1\%)$ & $2777s (13\%)$ & $17988s (86\%)$\\
	\hline
	$656k$ & $70035s$ & $202s (0\%)$ & $8499s (12\%)$ & $61334s (87\%)$\\
	\hline
	$868k$ & $131578s$ & $274s (0\%)$ & $15913s (12\%)$ & $115391s (87\%)$\\
	\hline
	$1161k$ & $234714s$ & $1212s (1\%)$ & $25480s (11\%)$ & $208022s (88\%)$\\
	\hline
	\end{tabular}
\caption{Runtime Breakup of Individual Stages \cite{jschauer:thesis:2011}}
\label{tab:flowy2-profiling}
\end{table}

The runtime breakup of individual stages of the processing pipeline as shown
in \ref{tab:flowy2-profiling} reveal that the grouper and merger incur a
massive performance hit. A quick \marginpar{deep nested loops} investigation
hints towards usage of large deep nested loops in the merger with a worst-case
$O(n^3)$ runtime complexity.

In addition, pushing the flow-records data from one stage of the pipeline to
another involved deep copying of the whole flow data whereby a mere passing
across of a \marginpar{deep copy of flow records} reference across a pipeline
in a branch would have sufficed. Similar behavior is visible when the grouper
when passing group records saved the individual flow-records in a temporary
location tagged with the groups and/or subgroups they belonged to.

The decision decision to use PyTables to read and write flow-records in
\ac{HDF} format also added to the complexity. Since, the input flow-records
were most of the time \marginpar{pytables and hdf} in either
\texttt{flow-tools} or \texttt{nfdump} file-formats, each time they had to be
converted into \ac{HDF} file formats prior to Flowy's execution which was
unnecessary.

\section{Flowy Improvements}\label{sec:flowy-improvements}

The flow-querier parser written in \ac{PLY} and the validators written for
each stage of the processing pipeline that check for semantics correctness
were left unmodified, since their execution time was invariant of the size of
the input data and slightly varying on the query complexity in itself.

Thread affinity masks were set for each new thread created to delegate the
thread to a separate processor core. \texttt{try/except} blocks were narrowed
down to only code that needed to be exception handled. A test-suite was
developed with few sample queries and input traces to validate Flowy's results
for regression analysis. A \texttt{setup.py} script was written to facilitate
installation of \marginpar{affinity masks, easier installation and
configuration, better profiling and testing, extended command line switches}
Flowy and its dependencies and \texttt{options.py} was replaced with
\texttt{flowy.conf} configuration file with the standard human-readable
key-value pairs. The command line option handling was switched from
\texttt{optparse} to \texttt{argparse} module and a switch was added for easy
profiling. The profiling output was modified as well to allow standard tab
delimiters which can be easily parsed by other tools. The flow query was also
extended to allow file contents to be supplied using \texttt{stdin}. Variable
names that are now part of Python identifiers were renamed.

A C library was written to parse and read/write flow-records in
\texttt{flow-tools} compatible format. The C library was connected to the
Python prototype using Cython \cite{dseljebotn:2009}\cite{wilbers:2009}. This
allowed the flow-records to be easily referenced by an \marginpar{cython to
connect c extensions to python} identifier, thereby giving away the need to
every time copy all the flow-records when moving ahead in the processing
pipeline. Cython was used since it allowed to write C extensions in a Pythonic
way by strong-typing variables, calling native C libraries and allowing usage
of pointers and structs, thereby providing the best of both worlds
\cite{sbehnel:2011}.

A custom C library was written to directly read/write data in the
\texttt{flow-tools} format to provide a drop-in replacement for PyTables and
overcome the overhead of format conversions. The library sequentially reads
the complete flow-records into memory to support random access required for
relative filtering. \marginpar{a custom c library to replace pytables} Each
flow-record is stored in a \texttt{char} array and the offsets to each field
are stored in a separate \texttt{struct}. The array of such records are
indexed allowing fast retrieval in $O(1)$ time. The C library is currently
limited to support \emph{only} \texttt{flow-tools} formats; \texttt{nfdump}
file formats are yet to be supported.

A design decision was made to rewrite the entire processing pipeline in C.
However, currently the core cannot parse the flow-query file, therefore the
execution is triggered by a tedious manual filling of the \texttt{structs} by
the contents of the query.

A filter stage \texttt{struct} is shown in listing \ref{lst:filterrule}. The
field to be filtered is indicated using a \texttt{field\_offset} and
\texttt{field\_length} in the \texttt{char} array of a records. The value to
be compared against with is also supplied which can be either a
\marginpar{filter stage struct} static value or another field of a record.
\texttt{func} is a function pointer to the operation that is to be carried out
on a record whose record identifier is passed to it. The filter runs in $O(n)$
time as it needs to traverse through all the records of the \texttt{char}
array.

\lstset{caption=Filter Rule Struct \cite{jschauer:thesis:2011},
				basicstyle=\tiny\ttfamily, tabsize=2, language=C, numbers=left,
        stepnumber=1, numberstyle=\ttfamily\color{gray},
        keywordstyle=\color{blue}, frame=shadowbox, rulesepcolor=\color{black},
			  label=lst:filterrule, aboveskip=20pt, captionpos=b}
\begin{lstlisting}
struct filter_rule {
	size_t field_offset;
	uint64_t value;
	uint64_t delta;
	bool (*func)(
		char *record,
		size_t field_offset,
		uint64_t value,
		uint64_t delta);
};
\end{lstlisting}

Similarly, a merger stage \texttt{struct} is shown in listing
\ref{lst:mergerrrule}. \texttt{branch\{1,2\}} are branch identifiers and
\texttt{field\{1,2\}} are the aggregated field identifers in the order of
aggregation. \texttt{func} is a function pointer pointing to the operation to
be carried out. The merger \marginpar{merger stage struct} runs in $O(n^k)$
time where $k$ is the number of branches. The \texttt{char} arrays in each
branch are disjoint since a record cannot be part of more than one group.

\lstset{caption=Merger Rule Struct \cite{jschauer:thesis:2011},
				basicstyle=\tiny\ttfamily, tabsize=2, language=C, numbers=left,
        stepnumber=1, numberstyle=\ttfamily\color{gray},
        keywordstyle=\color{blue}, frame=shadowbox, rulesepcolor=\color{black},
			  label=lst:mergerrrule, aboveskip=20pt, captionpos=b}
\begin{lstlisting}
struct merger_rule {
	size_t branch1;
	size_t field1;
	size_t branch2;
	size_t field2;
	uint64_t delta;
	bool (*func)(struct group *group1,
		size_t field1,
		struct group *group2,
		size_t field2,
		uint64_t delta);
};
\end{lstlisting}

The current core implementation also strictly adheres to the processing
pipeline shown in figure \ref{fig:flowy-pipeline}. As such, it is not
currently possible to skip \marginpar{core limitations} stages. In addition it
is not currently possible to have more than one merger or grouper in the
flow-query or aggregate fields in the grouper module since \texttt{char} array
storage is not possible.


\section{Benchmarks}\label{sec:benchmarks}

\begin{table}[h!]
	\begin{center}
		\begin{tabular}{|c|c|c|}
		\hline
		Number of records & Flowy & Flowy $2.0$ \\
		\hline
		\hline
		$103k$ & $1177s$ & $0.3s$\\
		\hline
		$337k$ & $20875s$ & $3.4s$\\
		\hline
		$656k$ & $70035s$ & $13s$\\
		\hline
		$868k$ & $131578s$ & $23s$\\
		\hline
		$1161k$ & $234714s$ & $86s$\\
		\hline
		\end{tabular}
	\end{center}
\caption{Flowy vs Flowy2 \cite{jschauer:thesis:2011}}
\label{tab:flowy2-benchmarks}
\end{table}

A flow query with the union aggregations stripped off was used as a sample to
compare the runtime performance of Flowy \cite{kkanev:thesis:2009} with Flowy
$2.0$ \marginpar{flowy 2.0 vs flowy} \cite{jschauer:thesis:2011}. The
benchmarks are shown in figure \ref{tab:flowy2-benchmarks}. It is conspicuous
how well the replacement of the core algorithms from Python to C turned out to
be.

In another test, Flowy $2.0$'s functionality was reduced to absolute filtering
\marginpar{flowy 2.0 vs flow-tools} to compare its performance with a
state-of-the-art \texttt{flow-tools} analysis tool using
listing \ref{lst:flowy2vflowtools}. It turned out Flowy $2.0$ performed just as
comparable if not better on an average.

\lstset{caption=Flowy2 vs \texttt{flow-tools} \cite{jschauer:thesis:2011},
				basicstyle=\tiny\ttfamily, tabsize=2, keywordstyle=\color{blue},
        language=bash, frame=shadowbox, rulesepcolor=\color{black},
			  label=lst:flowy2vflowtools, aboveskip=20pt, captionpos=b,
				moredelim=[s][\ttfamily]{"}{"}}
\begin{lstlisting}
$ time sh -c "flow-cat traces | flow-filter -P80"
$ time sh -c "flow-cat traces | ./flowy"
\end{lstlisting}


\section{Future Outlook}\label{sec:flowy2-future}

In a follow up to a commendable effort in making the Flowy prototype
drastically improve by orders of magnitude, the author in
\cite{jschauer:thesis:2011} has suggested numerous areas of improvement to
make the software fully functional again.

The Python prototype is currently left unused. The idea is at this stage is to
allow the Python prototype to parse and validate the flow query file which in
turn would pass the contents to a Cython wrapper which on the fly will forward
them \marginpar{system integration} to the core to properly fill in the
structs. At this point, the C core will process the query pipeline and
eventually forward back the results to the Python prototype which it can use
to display the results in a human friendly format.

The benchmarks performed in \cite{jschauer:thesis:2011} had a complexity of
$O(n^2)$ for the grouper and merger. This was when the number of branches in
the pipeline was reduced to maximum of $2$ and the flow-query had a single
module for both the merger and grouper. With the current implementation, this
complexity is deemed to increase exponentially as the number of records,
branches and the grouper, merger modules in the flow-query increase.
\marginpar{searching with trees} Therefore, having a search tree lookup for
the grouper and merger stage would help bring the runtime costs down, whereby
one of the fields will be traversed sequentially in $O(n)$ time and for each
field comparison will be performed by search tree lookups in $O(log(n))$ time
bringing down the complexity to $O(nlog(n))$. B+trees would essentially work
in this case, since records can still be traversed sequentially along a list
after a search tree lookup.

The comparison operations are currently passed an offset and the length of the
field type to be compared as shown in listings \ref{lst:filterrule},
\ref{lst:mergerrrule}. The length needs to be checked before making a cast to
an appropriate type inside these functions. Such checks can be avoided by
writing specialized functions for each combination of the field type $(33)$
and supported \marginpar{specialized functions in inner loops} operations
$(19)$ totaling to $20K$ functions. Such functions can be dynamically
generated from the Python code and would take around $3$MiB of space in memory
as suggested in \cite{jschauer:thesis:2011} which looks like worth the effort
considering these functions are invoked from the innermost loops in each stage
of the pipeline, and therefore squeezing such optimizations would go a long
way in improving the C core.

The core C implementation currently has limited multithreading. Each branch in
the pipeline runs on a separate thread and uses affinity masks to delegate the
thread to a separate processor core. However, this implies that merger and
ungrouper stages still remain single-threaded and the multithreaded
utilization largely depends on \marginpar{efficient multithreading} the query
being executed. The situation can be improved by writing a \texttt{pthreads}
wrapper that auto detects the number of available cores, creates a appropriate
size thread pool and equally divides the tasks among the threads in the pool.
This would also lead to increased complexity of managing mutual exclusion of
shared memory and needs to be investigated.
