%***********************************************************
\chapter{Flowy: Applications}\label{ch:flowy-applications}
%***********************************************************

The developed stream-based flow-querier helped to underpin a number of recent research efforts to solve real-world application problems that were deemed difficult before. This was possible due to the power and flexibility of the flow-query language to suit itself from generic to specific needs thereby opening doors of innovation. This section documents such efforts that use the in-house flow query language as well as a few others that exploit the flow level characteristics of the traffic patterns in general. 

\section{Application Identification using Flow 
Signatures}\label{sec:application-signatures}
The idea behind this study was to identify applications using flow traces on a network by analyzing potential left-behind signatures that describe them \cite{vperelman:2011, vperelman:thesis:2010}. This was based on the hypothesis that each application type generates unique flow signatures that might work as a fingerprint feature. To achieve this, a collection of network traces were recorded from several users and subsequently analyzed. The identified signatures were formalized by writing flow queries that were executed on Flowy \cite{kkanev:2010}. Several separate instances of the network traces were queried to evaluate the approach and come to a conclusion. 

\lstset{caption=Skype Application Signature \cite{vperelman:thesis:2010}, 
				tabsize=2, language=C, numbers=left,stepnumber=1,
				basicstyle=\tiny\ttfamily, numberstyle=\ttfamily\color{gray},
				keywordstyle=\color{blue}, frame=shadowbox, rulesepcolor=\color{black}, 	
			  label=lst:skypequery, aboveskip=20pt, captionpos=b}
\begin{lstlisting}
splitter S {}
	
...

merger M {
	module m1 {
		branches A, B
		A.srcip = B.srcip
		A o B OR B o A
	}
	export m1
}

ungrouper U {}
		
"input" -> S
S branch A -> F_SSDP -> G_SSDP -> GF_SSDP -> M
S branch B -> F_NAT_PMP -> G_NAT_PMP -> GF_NAT_PMP -> M
M -> U -> "output"
\end{lstlisting}
A formalized Flowy query to identify Skype from the flow traces for an instance is described in listing \ref{lst:skypequery}. The filter, grouper and group-filter sections of each branch are shown separately in listings \ref{lst:skypebranchA} and \ref{lst:skypebranchB}. Additional queries identifying variety of web browsers, mail clients, IM clients and media players can be found in \cite{vperelman:thesis:2010}.

\begin{center}
\begin{minipage}{.44\textwidth}
	\lstset{caption=Branch A \cite{vperelman:thesis:2010}, 
					tabsize=2, language=C, numbers=left,stepnumber=1,
					basicstyle=\tiny\ttfamily,numberstyle=\ttfamily\color{gray},
					keywordstyle=\color{blue},frame=shadowbox, rulesepcolor=\color{black}, 	
				  label=lst:skypebranchB, aboveskip=20pt, captionpos=b}
	\begin{lstlisting}
	filter F_SSDP {
		dstport = 1900
		port = protocol("UDP")	
		dstip = 239.255.255.250
	}

	grouper G_SSDP {
		module g1 {
			srcip = scrip
			dstip = dstip
			srcport = srcport
		}
		aggregate srcip, sum(bytes) as B
	}

	groupfilter GF_SSDP {
		B = 321
	}
	\end{lstlisting}
\end{minipage}
\hfill
\begin{minipage}{.44\textwidth}
	\lstset{caption=Branch B \cite{vperelman:thesis:2010}, 
					tabsize=2, language=C, numbers=left,stepnumber=1,
					basicstyle=\tiny\ttfamily, numberstyle=\ttfamily\color{gray},
					keywordstyle=\color{blue},frame=shadowbox, rulesepcolor=\color{black}, 	
				  label=lst:skypebranchA, aboveskip=20pt, captionpos=b}
	\begin{lstlisting}
	filter F_NAT_PMP {
		dstport = 5351
		port = protocol("UDP")	
	}

	grouper G_NAT_PMP {
		module g1 {
			srcip = scrip
			dstip = dstip
		}
		aggregate srcip, sum(bytes) as B
	}

	groupfilter GF_NAT_PMP {
		B = 160
	}
	\end{lstlisting}
\end{minipage}
\end{center}
The filter \texttt{F\_SSDP} is used to identify the four identical \ac{UDP} multicast messages the client sends out using \ac{SSDP} \cite{mbodlaender:2005}. Similarly \texttt{F\_NAT\_PMP} filter is used to identify four \ac{NAT-PMP} \cite{draft-cheshire-nat-pmp-03} messages sent over UDP. The groupers \texttt{G\_SSDP} and \texttt{G\_NAT\_PMP} group together flow records with the same source and destination IP address and the aggregate clauses describe the meta information with unique source IP addresses for each group records along with the total bytes carried within each group. The meta information is used to further filter the group-records in \texttt{GF\_SSDP} and \texttt{GF\_NAT\_PMP} modules. 

\begin{table}[h!]
	\begin{center}
		\tiny
		\begin{tabular}{|c|c|c|c|c|c|}
		\hline	
		UserID & Skype & Opera & Amarok & Chrome & Live \\
		\hline
		\hline 
		u$0$ & \ding{52} & \ding{109} & \ding{54} & \ding{109} & \ding{109} \\
		\hline             
		u$1$ & \ding{52} & \ding{109} & \ding{109} & \ding{109} & \ding{109} \\
		\hline             
		u$2$ & \ding{109} & \ding{109} & \ding{109} & \ding{109} & \ding{109} \\
		\hline             
		u$3$ & \ding{52} & \ding{109} & \ding{54} & \ding{109} & \ding{109} \\
		\hline             
		u$4$ & \ding{109} & \ding{109} & \ding{109} & \ding{109} & \ding{109} \\
		\hline             
		u$5$ & \ding{52} & \ding{109} & \ding{52} & \ding{52} & \ding{109} \\
		\hline             
		u$6$ & \ding{109} & \ding{109} & \ding{109} & \ding{109} & \ding{109} \\
		\hline
		u$7$ & \ding{109} & \ding{52} & \ding{52} & \ding{109} & \ding{109} \\
		\hline
		u$8$ & \ding{109} & \ding{109} & \ding{109} & \ding{109} & \ding{109} \\
		\hline
		u$9$ & \ding{52} & \ding{52} & \ding{52} & \ding{52} & \ding{109} \\
		\hline
		\end{tabular}
	\end{center}
\caption{Application Flow Signatures: Results \cite{vperelman:thesis:2010}}
\label{tab:flow-sig-results}
\end{table}
The identification results obtained from the analysis of flow-traces from ten unique users are compiled together in table \ref{tab:flow-sig-results}. The results demonstrate a success rate of $96\%$ for the five applications tested.  This study reveals that it is possible to identify applications from their network flow fingerprints and is a first step towards automating the complete process whereby machine learning techniques would be used to automatically generate flow-queries and identify new applications and even more so newer versions of the same application. 

\section{Cybermetrics: User Identification}\label{sec:cybermetrics}

The idea of identification of users based on biometric patterns such as keystroke dynamics \cite{fbergadano:2002}, mouse interactions \cite{aahmed:2007} or activity cycles in online games \cite{kchen:2007} has been long known. This study takes the idea even further by using flow-record patterns as a characteristic (cybermetrics) to identify a user on a network \cite{nmelnikov:thesis:2010, nmelnikov:2010}. Such a cybermetric user identification can be used for the purpose of providing secure access, system administration and network management. The feature extraction module of the analyzer as shown in figure \ref{fig:cybermetrics-overview} uses three distinct feature sets that could possibly be used to identify a user from a flow-record trace. 
\begin{figure}[h!]
\begin{center}
  \includegraphics* [width=0.7\linewidth]{figures/cybermetrics-overview}	
  \caption{Cybermetrics: Overview \cite{nmelnikov:thesis:2010}}
  \label{fig:cybermetrics-overview}
\end{center}
\end{figure}

Initial research efforts started with identifying application signatures in flow-records in \cite{vperelman:2011, vperelman:thesis:2010} and became relevant because different people have different preferences in the applications they use \marginpar{application signatures} and as such a set of applications in flow-records is a characteristic feature of a user. Flowy queries were formalized for four different set of applications and tested against a known set of users. The evaluation results of the derived queries as shown in table \ref{tab:flow-sig-results} demonstrated a strong evidence of presence (or absence) of applications and thereby provided an eventual marker for user identification. 

\begin{figure}[h!]
\begin{center}
  \includegraphics* [width=0.60\linewidth]{figures/cybermetrics-geography}	
  \caption{Geographical Preferences \cite{nmelnikov:thesis:2010}}
  \label{fig:cybermetrics-geography}
\end{center}
\end{figure}
The authors also looked into the geographical affiliations of different users by analyzing the \ac{ccTLD} of the browsed websites. They proposed a hypothesis that a user's origins strongly \marginpar{geographical preferences} influences their browsing activity. The analysis of the results established that the top five visited \ac{ccTLD}s constituted more than $85\%$ of the overall number of a user's visits as shown in figure \ref{fig:cybermetrics-geography}.
 
\begin{figure}[h!]
\begin{center}
  \includegraphics* [width=0.6\linewidth]{figures/cybermetrics-daily-distrib}	
  \caption{Daily Distributions for HTTP Traffic \cite{nmelnikov:thesis:2010}}
  \label{fig:cybermetrics-daily-distrib}
\end{center}
\end{figure}
In the end, the authors introduced a proof-of-concept method of user differentiation based on statistical features. These features considered daily distributions of parameters that were based on different port numbers. \marginpar{flow-record statistics}For instance, figure \ref{fig:cybermetrics-daily-distrib} shows the daily distribution of different users based on their \ac{HTTP} traffic usage. It was also witnessed that the time duration also played a key role in the process of feature formation, whereby the number of longer flows increased with the duration and consequently resulted in higher cross-correlations as shown in figure \ref{fig:cybermetrics-cross-correlate}  
\begin{figure}[h!]
\begin{center}
  \includegraphics* [width=0.6\linewidth]{figures/cybermetrics-cross-correlate}	
  \caption{Cross Correlation of Traces with Varying Times 
					 \cite{nmelnikov:thesis:2010}}
  \label{fig:cybermetrics-cross-correlate}
\end{center}
\end{figure}

This research is a first attempt to identify users based on their network flow fingerprints and the on-going effort is focussing on sophisticated machine learning techniques to learn behavioral patterns of known users to identify them in the future from their current network-flow traces.

\section{IPv$6$ Transition Failure Identification}\label{sec:ipv6transeval}
The IPv$4$ address space depletion is upon us and has become more imminent in the last few years. While IPv$6$ can readily expand the extent of the Internet, deploying it alone is clearly not a solution today and hence there are a continuum of transitioning solutions that would help in this migration. In this study \cite{vbajpai:2012} we evaluated the compatibility of popular applications with such transitioning solutions: NAT$64$ \cite{rfc6146} and Dual-Stack Lite \cite{rfc6333}. The goal was to find potential failures by identifying application failure signatures left behind in the flow-record traces using Flowy. These failure signatures could later be used by service providers to automate the detection and eventually shorten the deployment verification cycle.

\begin{figure}[h!]
\begin{center}
  \includegraphics* [width=0.8\linewidth]{figures/ipv6transeval-nat64-setup}	
  \caption{NAT$64$ Setup \cite{vbajpai:2012}}
  \label{fig:ipv6transeval-nat64-setup}
\end{center}
\end{figure}
In the NAT$64$ deployment testbed as shown in figure \ref{fig:ipv6transeval-nat64-setup}, the authors witnessed failure in $3$ applications: Skype, OpenVPN and Transmission. Flowy queries were defined to \marginpar{application operation under NAT$64$} establish failure signatures for each application. A formalized Flowy query to identify Skype failure signature for an instance is described in listing \ref{lst:skypefailure}. The filter sections of each branch are shown separately in listings \ref{lst:skypefailurebranchA} and \ref{lst:skypefailurebranchB}. 
\begin{center}
\begin{minipage}{.44\textwidth}
	\lstset{caption=Branch A \cite{vbajpai:2012}, 
					tabsize=2, language=C, numbers=left,stepnumber=1,
					basicstyle=\tiny\ttfamily,numberstyle=\ttfamily\color{gray},
					keywordstyle=\color{blue},frame=shadowbox, rulesepcolor=\color{black}, 	
				  label=lst:skypefailurebranchA, aboveskip=20pt, captionpos=b}
	\begin{lstlisting}
	filter f-mDNS {
	  dstport = 5353
	  srcport = 5353
	  dstip = 224.0.0.251
	  duration > 1 sec
	  duration < 5 sec
	}
	\end{lstlisting}
\end{minipage}
\hfill
\begin{minipage}{.44\textwidth}
	\lstset{caption=Branch B-C-D \cite{vbajpai:2012}, 
					tabsize=2, language=C, numbers=left,stepnumber=1,
					basicstyle=\tiny\ttfamily, numberstyle=\ttfamily\color{gray},
					keywordstyle=\color{blue},frame=shadowbox, rulesepcolor=\color{black}, 	
				  label=lst:skypefailurebranchB, aboveskip=20pt, captionpos=b}
	\begin{lstlisting}
	filter f-login1 {
	  dstport = 443
	  duration > 55 sec
	  duration < 59 sec
	}
	\end{lstlisting}
\end{minipage}
\end{center}

Filter \texttt{f-mDNS} is used to filter multicast messages used by Skype to discover clients in the link-local network sent to the destination IP address-port combination $(224.0.0.251:5353)$. Filter \texttt{f-login1} is used to filter $3$ unsuccessful attempts to contact the login server each in a separate branch. The \marginpar{skype failure signature} source port and the duration increases with decreasing number of packets for each subsequent flow. 
\lstset{caption=Skype Failure Signature \cite{vbajpai:2012}, 
				tabsize=2, language=C, numbers=left,stepnumber=1,
				basicstyle=\tiny\ttfamily, numberstyle=\ttfamily\color{gray},
				keywordstyle=\color{blue}, frame=shadowbox, rulesepcolor=\color{black}, 	
			  label=lst:skypefailure, aboveskip=20pt, captionpos=b}
\begin{lstlisting}
splitter S {}
	
...

grouper g-login1 {
  module g1 {
    srcport = srcport
    dstip = dstip
    dstport = dstport
  }     
  aggregate srcip, dstip, srcport, td,
  sum(packets) as pckt-sum, count(rec_id) as n
}

merger M {
  branches mDNS, LOGIN1, LOGIN2, LOGIN3
		      
  LOGIN1.srcip = LOGIN2.srcip
  LOGIN2.srcip = LOGIN3.srcip
  LOGIN1.dstip = LOGIN2.dstip
  LOGIN2.dstip = LOGIN3.dstip
	
  LOGIN1.srcport = LOGIN2.srcport rdelta 1
  LOGIN2.srcport = LOGIN3.srcport rdelta 1

  LOGIN1.pckt-sum > LOGIN2.pckt-sum
  LOGIN2.pckt-sum > LOGIN3.pckt-sum
	
  mDNS.td < LOGIN1.td
  mDNS.td < LOGIN2.td
  mDNS.td < LOGIN3.td
	
  mDNS < LOGIN1
  mDNS < LOGIN2
  mDNS < LOGIN3
}

"input"-> S
S br mDNS -> f-mDNS -> g-mDNS -> gf-mDNS -> M
S br LOGIN1 -> f-login1 -> g-login1 -> gf-login1 -> M
S br LOGIN2 -> f-login2 -> g-login2 -> gf-login2 -> M
S br LOGIN3 -> f-login3 -> g-login3 -> gf-login3 -> M
M -> U -> "output"

\end{lstlisting}
The groupers count the number of packets in each flow-records using \texttt{pckt-sum} which is later utilized by the merger stage to distinguish the branches. The group-filter stage finally is used to filter out groups with more than one record.

The NAT$64$ translation works when the applications running on the IPv$6$-only host explicitly make DNS requests to allow DNS$64$ to capture and masquerade them as fake IPv$6$ addresses that are eventually \marginpar{failure when using IPv$4$ literals}sent to the NAT$64$ box. If the applications use IPv$4$ literals to contact the servers, DNS$64$ is skipped and therefore NAT$64$ cannot perform the translation. This was reason behind the failure of the other two applications (OpenVPN and Transmission).

This study sets across a baseline to automate the failure detection by formalizing queries against flow-records. While a more exhaustive study encompassing wider set of applications still needs to be carried out, it is imperative that this unique approach is not just limited to IPv$6$ transition technologies, but can be utilized to identify failures in more generic cases.

\section{TCP level Spam Detection}\label{sec:spam-detection}
\section{OpenFlow}\label{sec:openflow}