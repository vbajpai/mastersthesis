%************************************************
\chapter{Future Work and Conclusion}\label{ch:future-work}
%************************************************

The C execution engine is a major leap forward to a fast and robust
implementation of \ac{NFQL}. It is the first time when a \ac{NFQL}
implementation can actually be used on practically sized flow traces. However
it is far from being deemed complete. The fast-paced development is clear from
the fact that the engine's issue tracker
\footnote{\url{https://github.com/vbajpai/mthesis-src/issues}} has a $1:1$
ratio of closed to open issues. As a result, the future outlook of the engine
is divided into major goals and minor issues that still need to be addressed.

\section{Major Goals}\label{sec:major-goals}

The execution engine uses the \texttt{flow-tools} \ac{API} to read and parse
the flow-records. The \texttt{flow-tools} \ac{API} can only parse NetFlow $v5$
records, thereby sandboxing the engine's functionality. It restricts the
engine's understanding of a flow to a fixed NetFlow $v5$ format and inhibits
the capability to parse \aci{IP}$v6$ flows. Since NetFlow $v9$ and now with
\ac{IPFIX}, the flow format \marginpar{ipfix support} is dynamically parsed
using its accompanied template. It relaxes the definition of a flow and gives
more power to the sender on how the data should be conglomerated together.
There are three well-known implementations of \ac{IPFIX}: CERT'
\texttt{libfixbuf} \footnote{\url{http://tools.netsa.cert.org/fixbuf/}},
Fraunhofer FOKUS' \texttt{libipfix}
\footnote{\href{http://libipfix.git.sourceforge.net/git/gitweb.cgi?p=libipfix/libipfix;a=summary}{\texttt{http://libipfix.git.sourceforge.net/}}}
and WAND' \texttt{Maji}
\footnote{\url{http://research.wand.net.nz/software/maji.php}}. The former two
\ac{API}s appear to be under heavy development and one of them is foreseen to
be used by the engine to provide \ac{IPFIX} capability in the future.




The python parser is currently unused. The idea at this stage is to allow the
parser to parse and validate the flow query and generate an equivalent
intermediate \texttt{JSON} format. This is supposed to be a preprocessing
step. The \texttt{JSON} flowquery file can then later be supplied to the
execution engine at runtime.  It can also be foreseen to push parts of the
\texttt{JSON} flowquery using a RESTful interface to \marginpar{a frontend
parser} multiple map/reduce jobs running the execution engine to completely
distribute the workflow for faster processing. The first step to achieve such
a convergence will be by removing the processing pipeline implementation code
out of the python framework. The reverse engineered package and class diagrams
generated using \texttt{pyreverse} are available in \texttt{parser/docs/} and
will help one get started with this task. The installation and usage
instructions available in the appendix will make the convergence head start a
breeze. Alternatively, it is also possible to re-implement the frontend parser
from scratch using \texttt{lex/YACC}.




\lstset{caption=F$(v2):$ GNU gprof on Filter,
				tabsize=2, language=shell, numbers=left,stepnumber=1,
        basicstyle=\tiny\ttfamily, numberstyle=\ttfamily\color{gray},
        keywordstyle=\color{blue}, frame=shadowbox,
        rulesepcolor=\color{black}, label=lst:fv2-gprof-filter,
        aboveskip=20pt, captionpos=b}
\begin{lstlisting}
$ bin/engine examples/filter/filter-1.0.json examples/trace-2012.ftz

   % time   	name
   66.67  ... ft_read
   33.33  ... ftio_read
   ...
\end{lstlisting}

The execution engine currently memory maps the (usually small) \texttt{JSON}
query file. However, the (usually large) trace file is not memory mapped. This
is because the engine uses the \texttt{flowtools} library which uses
\texttt{read} system calls to read the trace. Listing
\ref{lst:fv2-gprof-filter} shows the GNU \texttt{gprof} \cite{graham:1982}
profiling results when \marginpar{memory mapping the tracefile, multithreaded
filtering} stressing only the filter stage.  It is clear that the majority of
the time is taken in reading the flow records from the trace file.  Memory
mapping the trace file to engine's virtual space will increase I/O
performance. This is because the \texttt{mmap} region is a kernel's cache
itself avoiding the need to create further copies in userspace.  In addition,
accessing the engine's virtual space is in orders of magnitude faster to
making system calls and a linear walk through the file only requires disk
access across page boundaries. Memory mapping will also allow to parallelize
the filter stage by taking advantage of the available random file access.





The execution engine uses the \texttt{flowtools} API to write the results as
NetFlow v$5$ records. Writing the results to flat files is important since it
allows one to not only blackbox a pipeline stage operation, but also help plug
the results to another network analysis tool, thereby increasing
interoperability.\marginpar{lzo compression support} The \texttt{flowtools}
API, however, currently only support \texttt{zlib} compression. Contemporary
tools like \texttt{nfdump} and SiLK generally use
\texttt{lzo}\footnote{\url{http://www.oberhumer.com/opensource/lzo/}}
compression to achieve faster compression and decompression rates when doing
I/O operations.  Adding \texttt{lzo} compression support in the
\texttt{flowtools} API will further decrease the flow read/write times of the
execution engine.




\lstset{caption=F$(v2):$ GNU gprof on Grouper,
				tabsize=2, language=shell, numbers=left,stepnumber=1,
        basicstyle=\tiny\ttfamily, numberstyle=\ttfamily\color{gray},
        keywordstyle=\color{blue}, frame=shadowbox,
        rulesepcolor=\color{black}, label=lst:fv2-gprof-grouper,
        aboveskip=20pt, captionpos=b}
\begin{lstlisting}
$ bin/engine examples/grouper/grouper-1.0.json examples/trace-2012.ftz

   % time   	name
   40.00  ... comp_uint32_t
   12.73  ... qsort_comp
    7.27  ... comp_uint16_t
    5.45  ... comp_uint8_t
    ...
\end{lstlisting}

The grouper in F$(v2)$ sorts the filtered recordset on all the grouping rules.
This implies that the comparator has to make nested calls using all the rules
until a \marginpar{parallelized grouper qsort} tie-breakup is resolved. This
allows the grouper to make a nested \texttt{bsearch} invocation for the
generic case to reduce a linear pass through the filtered recordset.  However,
the number of comparator calls increase and now account for almost $60\%$ of
the time taken in the grouper as can be seen from the GNU \texttt{gprof}
results shown in listing \ref{lst:fv2-gprof-grouper}. The \texttt{qsort} call
is currently single-threaded. Since the size of the filtered recordset is
known before the invocation, bifurcating the \texttt{qsort} invocation to
multiple threads and merging the results back using merge sort will help
parallelize the operation.





 The current grouper implementation runs in $O(p * n*lg(n)) + O(n) +
 O(n*lg(k))$, while the merger runs in $O(n^m)$ time, where $n$ is the number
 of flow records, $m$ is the number of branches, $k$ is the number of unique
 \marginpar{search tree and hash table lookups} filtered records and $p$ is
 the number of grouper terms.  Therefore, a search tree lookup would help
 bring the runtime costs down, whereby one of the fields will be traversed
 sequentially in $O(n)$ time and for each field, a comparison will be
 performed by search tree lookups in $O(log(n))$ time bringing down the
 overall complexity of both merger and grouper to $O(nlog(n))$ respectively.
 In addition, letting the execution engine override the search tree lookups by
 hash table lookups for equality operators will further bring down the runtime
 to $O(n)$ for this specific case.











The execution engine currently has limited multithreading. Each branch in the
pipeline runs on a separate thread. However, this implies that the merger and
ungrouper stages still remain single-threaded. It is possible to handle the
merger's outermost branch loop using multiple threads in a non-blocking
fashion to improve performance \marginpar{multithreaded merger}, or by
writing a \texttt{pthreads} wrapper that auto detects the number of available
cores, creates a appropriate size thread pool and equally divides the tasks
among the threads. This would also lead to an increased complexity of managing
mutual-exclusion of shared memory, but the performance gains will go a long
way.






F$(v2)$ introduces a regression test-suite framework that helps keep the
quality of the execution engine at a certain minimum threshold. However, the
execution engine is as robust as its test-suite. The tests defined in the
regression test-suite are not exhaustive and do not touch every aspect of the
engine's \marginpar{exhaustive regression test-suite} functionality. For
instance, the queries executed by the suite only hit few of many comparator
operations supported by the engine. Unless every aspect of the engine's offered
functionality has a test case associated with it, it cannot be guaranteed that
the engine will not fail in one or more corner cases not hit by the suite. As a
result, it is imperative that addition of more test-cases to touch these corner
cases will make the execution engine better





\section{Minor Issues}\label{sec:minor-issues}




The execution engine in F$(v2)$ introduces two approaches to boost up the speed
of the grouper. One of the approaches uses \texttt{qsort} on each requested
grouping rule and a nested \texttt{bsearch} to find a more specific filtered
recordset pointer to signicantly reduce the linear pass when grouping members.
This approach works well for a generic case and spans different type of
comparator operations. This approach is further optimized when the grouping is
requested on equality comparison. In such a scenario, the
\marginpar{dynamically choosing an appropriate grouper approach} need to
perform a \texttt{bsearch} goes away, and groups can be formed in a single
linear pass on the filtered recordset. The engine currently uses the second
approach, while the first approach is in the source code history. Ideally, the
engine must be smart to use the second approach when equality grouping is
requested, and fallback to the first approach otherwise.




The group filter is currently a separate module in the execution engine as
shown in listing \ref{lst:fv2-groupfilter-indy}. This means that it is invoked
only after the grouper returns. As a consequence, the group filter has to
iterate over the groups again in order to make decision on which ones to filter
out. This decision can also be m\marginpar{inline group filter} made as soon as
the group is realized in the grouper. Such a refactor will phase out the
distinct boundary between the grouper and group filter in the source code.
However, it will eliminate a duplicate iteration of the groupset in the group
filter thereby improving performance. The performance gains will be more
visible when the ratio of the number of groups formed to the number of input
filtered records is high.

 \lstset{caption=F$(v2):$ Group Filter as an Independent Module,
				tabsize=2, language=c, numbers=left,stepnumber=1,
        basicstyle=\tiny\ttfamily, numberstyle=\ttfamily\color{gray},
        keywordstyle=\color{blue}, frame=shadowbox,
        rulesepcolor=\color{black}, label=lst:fv2-groupfilter-indy,
        aboveskip=20pt, captionpos=b}
\begin{lstlisting}
if (groupfilter_enabled) {

  branch->gfilter_result = groupfilter(
                                        branch->num_groupfilter_clauses,
                                        branch->groupfilter_clauseset,

                                        branch->grouper_result,

                                        branch->data,
                                        branch->branch_id
                                      );
  ...
}
\end{lstlisting}

The \texttt{START} and \texttt{END} timestamps of the cooked NetFlow $v5$
group records are currently unset. These timestamps should be superset of
their member \marginpar{superset intervals in group records} flow-records.
This will allow the group filter to skip checking the timestamps of each
member of the group and only focus on the superset interval of the group
record itself when performing allen interval operations and is an easy issue
to resolve.

\lstset{caption=F$(v2)$: Redundant Structs,
				tabsize=2, language=C, numbers=left,stepnumber=1,
        basicstyle=\tiny\ttfamily, numberstyle=\ttfamily\color{gray},
        keywordstyle=\color{blue}, frame=shadowbox,
        rulesepcolor=\color{black}, label=lst:fv2-redundant-structs,
        aboveskip=20pt, captionpos=b}
\begin{lstlisting}
struct json*
json_query = parse_json_query(param_data->query_mmap);
struct flowquery*
fquery = prepare_flowquery(param_data->trace, json_query);
\end{lstlisting}

The execution engine currently uses an additional data structure
(\texttt{struct json}) \marginpar{eliminating redundant structs} to hold the
parsed \texttt{JSON} query. This data structure is then read by
\texttt{prepare\_flowquery(\ldots)} to generate \texttt{struct flowquery} which
is eventually used by the pipeline stages as shown in listing
\ref{lst:fv2-redundant-structs}. In essence, the intermediate \texttt{struct}
is not needed, and is a redundant datastructure. There is no reason why
\texttt{parse\_json\_query(...)} cannot directly read the query elements into
\texttt{struct flowquery} and is a future refactor item.

The ungrouper reads the tuples of merged group records and unfolds each one of
them to create a stream. Each such stream is a collection of flow-records that
\marginpar{eliminating redundant flows from a stream} passed the whole
pipeline. However, it is possible that a flow record is part of multiple
groups in a single group tuple, and is therefore outputted multiple times. The
engine currently does not eliminate such flow records repetitions. In the
future, the engine can take an option if one desires to eliminate such
repetitions. It is also does not order the flow records according to their
timestamps as defined in the \ac{NFQL} specification.


\section{Conclusion}\label{sec:conclusion}

The \ac{NFQL} execution engine has come a long way in a short time. It now
consists of a robust implementation of the processing pipeline that adapts
itself to the kind of query provided at runtime to dynamically decide the type
of data and the type of operation to be performed. It is flexible to be able
to read and parse an entire flowquery at runtime. It is fast to be able to
process millions of flow traces in matter of seconds. It is portable and can
seamlessly build on multiple Unix flavors and is verifiable using a regression
test-suite that will allow future developers to work further to improve the
engine with confidence.
