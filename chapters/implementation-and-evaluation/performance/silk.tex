The benchmarking suite was used to run a number of queries over a public flow
trace containing $20M$ records. We used trace $#7$
\footnote{\url{http://traces.simpleweb.org/traces/netflow/netflow1/netflow000.tar.bz2}},
from Simpleweb \footnote{Simpleweb is a data repository of traffic traces from
University of Twente}.  The input trace was compressed at \texttt{ZLIB\_LEVEL}
$5$ using the \texttt{zlib} suite. It was also converted to \texttt{nfdump}
and SiLK\footnote{Detailed conversion instructions are available in the
appendix} \cite{SiLK} compatible formats and compressed with \texttt{zlib}
keeping the same compression level.  The suite was run on a high-end
machine\footnote{\url{crystal.eecs.jacobs-university.de}} with $24$ cores of
$2.5$ GHz clock speed and $18$ GiB of memory. The results and graphs are
available on the \texttt{benchmarks} branch of the \texttt{git} repository.

\lstset{caption=Filter Stage: SiLK Queries,
				tabsize=2, language=bash, numbers=left,stepnumber=1,
        basicstyle=\tiny\ttfamily, numberstyle=\ttfamily\color{gray},
        keywordstyle=\color{blue}, frame=shadowbox,
        rulesepcolor=\color{black}, label=lst:silk-filter,
        aboveskip=20pt, captionpos=b, upquote=true}
\begin{lstlisting}
# ratio: 0.0
rm -f $OUTPUT; \
rwfilter --packets=0-  --compression-method=zlib --fail=$OUTPUT $INPUT;

# ratio: 0.2
rm -f $OUTPUT; \
rwfilter --packets=3- --compression-method=zlib --pass=$OUTPUT $INPUT;

# ratio: 0.4
rm -f $OUTPUT; \
rwfilter --packets=2- --compression-method=zlib --pass=$OUTPUT $INPUT;

# ratio: 0.6
rm -f $OUTPUT; \
rwfilter --packets=2- --compression-method=zlib --fail=$OUTPUT $INPUT;

# ratio: 0.8
rm -f $OUTPUT; \
rwfilter --packets=3- --compression-method=zlib --fail=$OUTPUT $INPUT;

# ratio: 1.0
rm -f $OUTPUT; \
rwfilter --packets=0- --compression-method=zlib --pass=$OUTPUT $INPUT;
\end{lstlisting}

The first query attempts to stress the filter stage.  It uses varying values
on the packet field offset to determine the amount of flow records that are
passed by the filter. The resultant filtered records are written to
\texttt{flow-tools} compatible file format and compressed at  at
\texttt{ZLIB\_LEVEL} $5$ \marginpar{stressing the filter} using \texttt{zlib}
suite. The ratio of the number of filtered records in the output trace to the
number of the flow records in the input trace is plotted against time. SiLK
example queries using \texttt{rwfilter} are shown in listing
\ref{lst:silk-filter}. The queries for \ac{NFQL}, \texttt{flowtools},
\texttt{nfdump} are similar and can be referenced from \texttt{benchmarks}
branch\footnote{\texttt{benchmarks/august/filter/queries/\{flowtools, nfql,
nfdump, silk\}}}. The evaluation results are shown in figure
\ref{fig:benchmarks-filter}.

\begin{figure}[h!]
  \begin{center}
    \includegraphics* [width=0.8\linewidth]{figures/benchmarks/filter}
    \caption{Filter Stage: NFQL vs SiLK, Flow-Tools, Nfdump}
    \label{fig:benchmarks-filter}
  \end{center}
\end{figure}

It can be seen that the performance of the filter stage in \ac{NFQL} is
comparable to that of \texttt{flowtools} and SiLK. SiLK takes less time on
lower ratios, but then again SilK and \texttt{nfdump} also use their own
proprietary format for trace files. As a result, the amount of data that needs
to read (or written) may be different to what it is for \ac{NFQL} and
\texttt{flowtools}. On the other hand, \texttt{nfdump} appears to be
significantly faster than the rest. This \marginpar{stressing the filter:
discussion} is because \texttt{nfdump} lacks \texttt{zlib} support, and as
such the files that were read and written used \texttt{lzo} compression scheme
which trades space for achieving faster compression and decompression. It is
important to note, that all the tools were single-threaded in this evaluation,
and did not completely utilize the $24$ cores that were at their disposal. It
comes as a realization, that filtering the input using multiple threads by
memory mapping the trace and adding \texttt{lzo} compression will drastically
improve \ac{NFQL}'s filter performance.

\lstset{caption=Grouper Stage: SiLK Queries,
				tabsize=2, language=bash, numbers=left,stepnumber=1,
        basicstyle=\tiny\ttfamily, numberstyle=\ttfamily\color{gray},
        keywordstyle=\color{blue}, frame=shadowbox,
        rulesepcolor=\color{black}, label=lst:silk-grouper,
        aboveskip=20pt, captionpos=b, upquote=true}
\begin{lstlisting}
# ratio: 0.1
rm -f /tmp/filter.rwz $OUTPUT; \
rwfilter --packets=0- --compression-method=zlib --pass=/tmp/filter.rwz $INPUT; \
rwsort --fields=sIP /tmp/filter.rwz | \
rwgroup --id-fields=sIP --summarize \
        --compression-method=zlib --output-path=$OUTPUT;

# ratio: 0.4
rm -f /tmp/filter.rwz $OUTPUT; \
rwfilter --packets=0- --compression-method=zlib --pass=/tmp/filter.rwz $INPUT; \
rwsort --fields=sIP,dIP,sPort,dPort /tmp/filter.rwz | \
rwgroup --id-fields=sIP,dIP,sPort,dPort --summarize \
        --compression-method=zlib --output-path=$OUTPUT;

# ratio: 0.8
rm -f /tmp/filter.rwz $OUTPUT; \
rwfilter --packets=0- --compression-method=zlib --pass=/tmp/filter.rwz $INPUT; \
rwsort --fields=5-9 /tmp/filter.rwz | \
rwgroup --id-fields=5-9 --summarize \
        --compression-method=zlib --output-path=$OUTPUT;

# ratio: 1.0
rm -f /tmp/filter.rwz $OUTPUT; \
rwfilter --packets=0- --compression-method=zlib --pass=/tmp/filter.rwz $INPUT; \
rwsort --fields=1-9 /tmp/filter.rwz | \
rwgroup --id-fields=1-9 --summarize \
        --compression-method=zlib --output-path=$OUTPUT;
\end{lstlisting}

The second query attempts to stress the grouper stage. It reuses the filter
query that produces a $1.0$ ratio to allow the grouper to receive the entire
trace as a filtered recordset. The grouper part of the query then gradually
increases the number of grouping terms in the \ac{DNF} expression to increase
the output/input ratio. The resultant groups \marginpar{stressing the grouper}
are again written as \texttt{flowtools} files using the same \texttt{zlib}
compression level. The ratio of the number of groups formed to the number of
the input filtered records is plotted against time. SiLK example queries using
a combination of \texttt{rwfilter-rwsort-rwgroup} are shown in listing
\ref{lst:silk-grouper}.  The queries for \ac{NFQL} are similar and can be
referenced from \texttt{benchmarks}
branch\footnote{\texttt{benchmarks/august/grouper/queries/\{nfql, silk\}}}.
\texttt{nfdump} and \texttt{flowtools} do not support grouping, and therefore
are not considered in this evaluation. The evaluation results are shown in
figure \ref{fig:benchmarks-grouper}.

\begin{figure}[h!]
  \begin{center}
    \includegraphics* [width=0.8\linewidth]{figures/benchmarks/grouper}
    \caption{Grouper Stage: NFQL vs SiLK}
    \label{fig:benchmarks-grouper}
  \end{center}
\end{figure}

The evaluation graph reveals that the performance of the \ac{NFQL} grouper
stage is close. The time taken by the tools are comparable on lower ratios,
but on higher ratios, \ac{NFQL} starts to drift apart. SiLK, however remains
almost linear throughout the evaluation. Since most of the time is taken in
writing the records to files, it is unclear whether SiLK's usage of a
proprietary format which may reduce reads/writes is responsible for the drift
on higher ratios. SiLK's \texttt{rwgroup} tool is also supplied a
\texttt{--summarize} flag in all the queries. This gives SiLK the leverage to
not store information about which members are part of the group.  \ac{NFQL} on
the other hand needs to allocate resources (which may take time) to keep this
information in its data structures, since \marginpar{stressing the grouper:
discussion} the ungrouper later may request to write the members of a group
while unfolding the tuples.  The ungrouper although was disabled in this
evaluation, the allocation of space for group members was not. It is also
important to note that both the tools again remained single-threaded
throughout the evaluation. SiLK took an advantage of an inherent concurency
arising from how the query is structured as one single \texttt{bash} command
using pipes. The pipe between \texttt{rwsort} and \texttt{rwgroup} makes the
two process run concurrently, the effect of which gets more pronounced on
higher ratios and can be a drift determining factor. The profiling results
from GNU \texttt{gprof} \cite{graham:1982} indicate that $60\%$ of the time is
taken in \texttt{qsort} comparator calls.  As a result, it comes as no
surprise, that bifurcating \texttt{qsort} invocation to multiple threads and
later merging the results back using merge sort will help parallelize the
grouper stage and maybe reduce the drift on higher ratios. In addition, since
all of the evaluation queries had grouping terms using an equality comparator,
\ac{NFQL} can introspect such a grouping rule to dynamically optimize
processing searches using a hashtable and turn to \texttt{qsort} based
grouping only as a fallback.


\lstset{caption=Group Filter Stage: SiLK Queries,
				tabsize=2, language=bash, numbers=left,stepnumber=1,
        basicstyle=\tiny\ttfamily, numberstyle=\ttfamily\color{gray},
        keywordstyle=\color{blue}, frame=shadowbox,
        rulesepcolor=\color{black}, label=lst:silk-mdns-dns-udp,
        aboveskip=20pt, captionpos=b, upquote=true}
\begin{lstlisting}
# ratio: 0.0
rm -f /tmp/filter.rwz /tmp/grouper.rwz $OUTPUT; \
rwfilter --packets=0- --compression-method=zlib --pass=/tmp/filter.rwz $INPUT; \
rwsort --fields=1-9 /tmp/filter.rwz | \
rwgroup --id-fields=1-9 --summarize \
        --compression-method=zlib --output-path=/tmp/grouper.rwz; \
rwfilter --packets=0- --compression-method=zlib /tmp/grouper.rwz \
         --fail=$OUTPUT;

# ratio: 0.2
rm -f /tmp/filter.rwz /tmp/grouper.rwz $OUTPUT; \
rwfilter --packets=0- --compression-method=zlib --pass=/tmp/filter.rwz $INPUT; \
rwsort --fields=1-9 /tmp/filter.rwz | \
rwgroup --id-fields=1-9 --summarize \
        --compression-method=zlib --output-path=/tmp/grouper.rwz; \
rwfilter --packets=3- --compression-method=zlib /tmp/grouper.rwz \
         --pass=$OUTPUT;

# ratio: 0.4
rm -f /tmp/filter.rwz /tmp/grouper.rwz $OUTPUT; \
rwfilter --packets=0- --compression-method=zlib --pass=/tmp/filter.rwz $INPUT; \
rwsort --fields=1-9 /tmp/filter.rwz | \
rwgroup --id-fields=1-9 --summarize \
        --compression-method=zlib --output-path=/tmp/grouper.rwz; \
rwfilter --packets=2- --compression-method=zlib /tmp/grouper.rwz \
         --pass=$OUTPUT;

# ratio: 0.6
rm -f /tmp/filter.rwz /tmp/grouper.rwz $OUTPUT; \
rwfilter --packets=0- --compression-method=zlib --pass=/tmp/filter.rwz $INPUT; \
rwsort --fields=1-9 /tmp/filter.rwz | \
rwgroup --id-fields=1-9 --summarize \
        --compression-method=zlib --output-path=/tmp/grouper.rwz; \
rwfilter --packets=2- --compression-method=zlib /tmp/grouper.rwz \
         --fail=$OUTPUT;

# ratio: 0.8
rm -f /tmp/filter.rwz /tmp/grouper.rwz $OUTPUT; \
rwfilter --packets=0- --compression-method=zlib --pass=/tmp/filter.rwz $INPUT; \
rwsort --fields=1-9 /tmp/filter.rwz | \
rwgroup --id-fields=1-9 --summarize \
        --compression-method=zlib --output-path=/tmp/grouper.rwz; \
rwfilter --packets=3- --compression-method=zlib /tmp/grouper.rwz \
         --fail=$OUTPUT;

# ratio: 1.0
rm -f /tmp/filter.rwz /tmp/grouper.rwz $OUTPUT; \
rwfilter --packets=0- --compression-method=zlib --pass=/tmp/filter.rwz $INPUT; \
rwsort --fields=1-9 /tmp/filter.rwz | \
rwgroup --id-fields=1-9 --summarize \
        --compression-method=zlib --output-path=/tmp/grouper.rwz; \
rwfilter --packets=0- --compression-method=zlib /tmp/grouper.rwz \
         --pass=$OUTPUT;
\end{lstlisting}



The third query attempts to stress the group filter stage. It reuses the
filter and grouper queries that produce a $1.0$ ratio to allow the group
filter to receive the entire trace as input. That means, each flow record of
the original trace now becomes a group record for the group filter.  The group
filter then reuses the same varying values of the packet offset to determine
the amount of groups that are filtered ahead. The filtered groups
\marginpar{stressing the group filter} are again written as \texttt{flowtools}
files using the same \texttt{zlib} compression level. The ratio of the number
of filtered groups formed to the number of the input group records is plotted
against time. SiLK example queries using a combination of
\texttt{rwfilter-rwsort-rwgroup-rwfilter} are shown in listing
\ref{lst:silk-grouper}.  The queries for \ac{NFQL} are similar and can be
referenced from \texttt{benchmarks}
branch\footnote{\texttt{benchmarks/august/groupfilter/queries/\{nfql,
silk\}}}.  The evaluation results are shown in figure
\ref{fig:benchmarks-grouper}.

\begin{figure}[h!]
  \begin{center}
    \includegraphics* [width=0.8\linewidth]{figures/benchmarks/groupfilter}
    \caption{Group Filter Stage: NFQL vs SiLK}
    \label{fig:fv1-fv2-filter}
  \end{center}
\end{figure}

It can be seen that the timings of \ac{NFQL} are far apart from that of SiLK.
It is due to the drift already created by the grouper at $1.0$ ratio in the
previous stage. As a result, the group filter comes into play only after $300$
seconds, whereas SiLK's group filtering already starts just below $150$
seconds. Even if \marginpar{stressing the groupfilter: discussion} we
normalize the graph, it can be observed that the group filter has a
significantly higher slope. This is because the group filter is a separate
blackbox in the execution engine. This means that it is only executed once the
grouper returns. As a result, the group filter has to reiterate the groups to
make a filtering decision, which costs time as is seen from the graph. The
group filter can also be merged with the grouper, so as to be able to make a
filtering decision as soon as a group is formed, thereby eliminating an extra
loop over.
