\lstset{caption=F$(v2)$: Automated Benchmarking,
				tabsize=2, language=bash, numbers=left,stepnumber=1,
        basicstyle=\tiny\ttfamily, numberstyle=\ttfamily\color{gray},
        keywordstyle=\color{blue}, frame=shadowbox,
        rulesepcolor=\color{black}, label=lst:fv2-automated-benchmarking,
        aboveskip=10pt, captionpos=b, upquote=true}
\begin{lstlisting}
[engine] $ make; sudo benchmarks/nfql.py bin/engine trace[s]/ querie[s]/
benchmarking nfql ...
executing: [engine tcp-session trace-2012]:  1 2 3 4 5 6 7 8 9 10 (3.315148 secs)
executing: [engine tcp-session trace-2009]:  1 2 3 4 5 6 7 8 9 10 (0.034624 secs)
...

[engine] $ sudo benchmarks/silk.py trace[s]/ querie[s]/
benchmarking silk ...
executing: [silk http-tcp-session trace-2009]:  1 2 3 4 5 6 7 8 9 10 (0.102465 secs)
executing: [silk http-tcp-session trace-2012]:  1 2 3 4 5 6 7 8 9 10 (0.279106 secs)
...
\end{lstlisting}

To be able to run and reproduce the benchmarking results as and when required
it was essential to automate the whole process. The target design was to be
able to use one script to run all sets of query-trace combination in one go
for each network analysis application as shown in listing
\ref{lst:fv2-automated-benchmarking}. The directories containing the traces
and the queries required by the script can be supplied as command line
arguments. Few examples are provided in \texttt{examples/}. The benchmarking
suite only runs on python $2.7$ and above. Attention is given to clear
\texttt{pagecaches}, \texttt{dentries} and \texttt{inodes} before each
iteration invocation as shown in listing \ref{lst:fv2-clear-caches}. The
script, therefore needs to run with \texttt{sudo} privileges. The results are
saved in \texttt{benchmarks/results/}. SiLK query files are simply bash
commands separated by a delimiter and are further discussed in the next
section.

\lstset{caption=F$(v2)$ Benchmarking: Clearing Kernel Caches,
				tabsize=2, language=python, numbers=left,stepnumber=1,
        basicstyle=\tiny\ttfamily, numberstyle=\ttfamily\color{gray},
        keywordstyle=\color{blue}, frame=shadowbox, stringstyle=\ttfamily,
        rulesepcolor=\color{black}, label=lst:fv2-clear-caches,
        aboveskip=20pt, captionpos=b, upquote=true}
\begin{lstlisting}
...
# clear pagecache, dentries and inodes
os.system('sync')
try:
  with open('/proc/sys/vm/drop_caches', 'w') as stream:
    stream.write('3\n')
...
\end{lstlisting}
