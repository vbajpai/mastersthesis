\lstset{caption=Merger Pseudocode \cite{vmarinov:2009},
				tabsize=2, language=C, numbers=left,stepnumber=1,
        basicstyle=\tiny\ttfamily, numberstyle=\ttfamily\color{gray},
        keywordstyle=\color{blue}, frame=shadowbox,
        rulesepcolor=\color{black}, label=lst:merger-pseudocode,
        aboveskip=20pt, captionpos=b}
\begin{lstlisting}
get_module_output_stream(module m) {
  (branch_1, branch_2, ..., branch_n) = get_input_branches(m);
  for each g_1 in group_records(branch_1)
    for each g_2 in group_records(branch_2)
      ...
        ...
           for each g_n in group_records(branch_n)
              if match(g_1, g_2, ..., g_n, rules(m))
                 output.add(g_1, g_2,..., g_n);
  return output;
}
\end{lstlisting}

The merger pseudocode as defined in the \ac{NFQL} specification
\cite{vmarinov:2009} is shown in listing \ref{lst:merger-pseudocode}.
Implementing this pseudocode in C is not straightforward. The level of nesting
depends on the number of branches, and is therefore not known at compile time.
The information on the number of branches comes from the query which is passed
to the execution engine at runtime.

\lstset{caption=F$(v2)$: Merger Iterator Utility,
				tabsize=2, language=C, numbers=left,stepnumber=1,
        basicstyle=\tiny\ttfamily, numberstyle=\ttfamily\color{gray},
        keywordstyle=\color{blue}, frame=shadowbox,
        rulesepcolor=\color{black}, label=lst:fv2-iter-codeblock,
        aboveskip=20pt, captionpos=b}
\begin{lstlisting}
/* initialize the iterator */
struct permut_iter *iter = iter_init(binfo_set, num_branches);

/* iterate over all permutations */
unsigned int index = 0;
while(iter_next(iter)) {
  index++
  for (int j = 0; j < num_branches; j++) {
    /* first item */
    if(j == 0)
      printf("\n%d: (%zu ", index, iter->filtered_group_tuple[j]);
    /* last item */
    else if(j == num_branches -1)
      printf("%zu)", iter->filtered_group_tuple[j]);
    else
      printf("%zu ", iter->filtered_group_tuple[j]);
  }
}

/* free the iterator */
iter_destroy(iter);
\end{lstlisting}

As a result, an iterator that can provide all possible permutations of
$N-$tuple (where $N$ is the number of branches) group record $ID$s was needed.
The result of the iterator can then be used to make a match. The merger stage,
therefore begins by initializing this iterator passing \marginpar{merger
iterator utility} it the number of branches, and information about each
branch. Then, it loops over to get a new $N-$tuple group record $ID$s on each
iteration until the iterator returns \texttt{false}. A sample to print all
possible group $ID$ permutation is shown in listing
\ref{lst:fv2-iter-codeblock}, with the output in listing
\ref{lst:fv2-iter-out}

\lstset{caption=F$(v2)$: Merger Iterator Utility Output,
				tabsize=2, language=C, numbers=left,stepnumber=1,
        basicstyle=\tiny\ttfamily, numberstyle=\ttfamily\color{gray},
        keywordstyle=\color{blue}, frame=shadowbox,
        rulesepcolor=\color{black}, label=lst:fv2-iter-out,
        aboveskip=20pt, captionpos=b}
\begin{lstlisting}
  1: (1 1 1)
  2: (1 1 2)
  ...
  12: (3 2 2)
\end{lstlisting}
