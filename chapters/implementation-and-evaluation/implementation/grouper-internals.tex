In order to be able to make comparisons on field offsets, a n\"aive approach
is to linearly walk through each filtered record against the filtered
recordset leading to a complexity of $O(n^2)$, where $n$ is the number of
filtered records. A smarter approach is to put the copy in a hash table and
then try to map each pointer while walking down the filtered recordset once,
leading to a complexity of $O(n)$. The hash table approach, although will work
on this specific example, will fail badly on other relative comparisons. The
F$(v1)$ \marginpar{functional grouping using qsort and bsearch} execution
engine \cite{jschauer:thesis:2011} formulates an approach to grouping records
using a binary search after a quick sort on the field offset of the first
grouping rule of each record in the filtered recordset. This helps the grouper
perform faster search lookups to find records that must group together in
$O(n*lg(k))$ time with a preprocessing step taking $O(n*lg(n)) + O(n)$ in the
average case, where $n$ is the number of filtered records and $k$ is the
number of unique filtered records.  The implementation of this idea was broken
and used to segfault at multiple steps within the grouping stage.  $F(v2)$
introduces a completely functional implementation of this approach that does
not incorporate any assumptions on the type of field offsets and is robust
enough to work with different types of grouping queries.

The reetrant \texttt{qsort\_r} was used, since it can pass an additional
argument \texttt{thunk} to the comparator, which in our case is the field
offset used for \marginpar{cross platform qsort\_r and bsearch\_r} comparing
two flow records. Since the order of arguments of \texttt{qsort\_r} are
different for \texttt{glibc} and \texttt{BSD}, the function invocation had to
be wrapped around platform specific macros as shown in listing
\ref{lst:fv2-qsortr}. More Surprisingly, there is currently no equivalent
\texttt{bsearch\_r} to complement \texttt{qsort\_r}. As such, the contemporary
\texttt{bsearch} function from the \texttt{glibc} library was adapted to
accommodate the void and is defined in \texttt{utils} module.

\lstset{caption=F$(v2)$: qsort\_r Invocation,
				tabsize=2, language=C, numbers=left,stepnumber=1,
        basicstyle=\tiny\ttfamily, numberstyle=\ttfamily\color{gray},
        keywordstyle=\color{blue}, frame=shadowbox,
        rulesepcolor=\color{black}, label=lst:fv2-qsortr,
        aboveskip=20pt, captionpos=b}
\begin{lstlisting}
struct grouper_type {
#if defined (__APPLE__) || defined (__FreeBSD__)
  int (*qsort_comp)(
                     void*            thunk,
                     const void*      e1,
                     const void*      e2
                   );
#elif defined (__linux)
  int (*qsort_comp)(
                     const void*      e1,
                     const void*      e2,
                     void*            thunk
                   );
#endif
...

#if defined(__APPLE__) || defined(__FreeBSD__)
  qsort_r(
           sorted_recordset_ref,
           num_filtered_records,
           get_grouper_intermediates(
           (void*)&grouper_ruleset[0]->field_offset2,
           gtype->qsort_comp
         );
#elif defined(__linux)
  qsort_r(
           sorted_recordset_ref,
           num_filtered_records,
           sizeof(char **),
           gtype->qsort_comp,
           (void*)&grouper_ruleset[0]->field_offset2
         );
#endif
\end{lstlisting}

Group records are a conglomeration of several flow records with some common
characteristics defined by the flow query. Some of the non-common
characteristics can also be aggregated into a single value using group
\marginpar{groups as cooked netflow v5 records} aggregations as shown in
listing \ref{lst:fv2-aggr-example}. Since, the execution engine supports
multiple verbosity levels, it is useful if a single group record can be again
mapped into a NetFlow $v5$ record template, so that it can be echoed as the
representative of all its members. This was achieved using a struct
\texttt{group} as shown in listing \ref{lst:fv2-group-struct}.

\lstset{caption=F$(v2)$: Group Struct,
				tabsize=2, language=C, numbers=left,stepnumber=1,
        basicstyle=\tiny\ttfamily, numberstyle=\ttfamily\color{gray},
        keywordstyle=\color{blue}, frame=shadowbox,
        rulesepcolor=\color{black}, label=lst:fv2-group-struct,
        aboveskip=20pt, captionpos=b}
\begin{lstlisting}
struct group {
  uint32_t                            num_members;
  char**                              members;
  struct aggr_result*                 aggr_result;
};

struct aggr_result {
  char*                               aggr_record;
  struct aggr**                       aggrset;
};

\end{lstlisting}

There can be a situtation where the query designer might incorrectly ask
for aggregation on a field already specified in a grouper (or filter)
module. If \marginpar{ignoring redundant aggregation requests} the
relative operator is an equality comparison, the aggregation on such a
field becomes less useful, since the members of the grouped record will
always have the same value for that field. The engine is now smart to
realize this redundant request and ignores such aggregations as shown in
listing \ref{lst:fv2-aggr-example}.

\lstset{caption=F$(v2)$: Aggregations Example,
				tabsize=2, language=C, numbers=left,stepnumber=1,
        basicstyle=\tiny\ttfamily, numberstyle=\ttfamily\color{gray},
        keywordstyle=\color{blue}, frame=shadowbox,
        rulesepcolor=\color{black}, label=lst:fv2-aggr-example,
        aboveskip=20pt, captionpos=b}
\begin{lstlisting}
grouper g_www_res {
   g1 {
      srcip = srcip
      dstip = dstip
   }
   aggregate srcip, dstip, sum(bytes) as bytes, bitOR(tcp_flags) as flags,
}

$ bin/engine queryfile tracefile --verbose=1
...
No. of Groups: ...

...    SrcIPaddress    ...      DstIPaddress      OR(Fl)    Sum(Octets)
...    4.23.48.126     ...      192.168.0.135     3         81034
...    8.12.214.126    ...      192.168.1.138     2         5065
\end{lstlisting}

Records are clubbed together into one group if no group modules are defined.
Previously such a query used to form groups for each individual filtered
record. \marginpar{clubbing records with no grouper rules} That was less
useful since then it was not possible to calculate meaningful aggregations on
all the records that passed the filter stage. Now, when the group modules are
empty, all the filtered records are clubbed into one group to allow
aggregations as shown in listing \ref{lst:fv2-club-example}.

\lstset{caption=F$(v2)$: Clubbing Records with No Grouper Rules,
				tabsize=2, language=C, numbers=left,stepnumber=1,
        basicstyle=\tiny\ttfamily, numberstyle=\ttfamily\color{gray},
        keywordstyle=\color{blue}, frame=shadowbox,
        rulesepcolor=\color{black}, label=lst:fv2-club-example,
        aboveskip=20pt, captionpos=b}
\begin{lstlisting}
grouper g_www_res {
   g1 {}
   aggregate sum(bytes) as bytes
}

$ bin/engine queryfile tracefile --verbose=1
...
No. of Groups: 1 (Aggregations)

...    Sum(Octets)
...    2356654
\end{lstlisting}

