Reverting to a flat source structure.

\begin{lstlisting}
>>> import this
The Zen of Python, by Tim Peters

…
Flat is better than nested.
…
\end{lstlisting}

The previous source structure was unnecesarily nested for the simple
number of source files that we have.\\Switching to a flat source
structure reduced the CMake compilation complexity alot. This is how it
looks like now:

\begin{lstlisting}
[engine] >> tree
.
|-- examples/
|   `-- trace.ft
|-- include/
|   |-- base.h
|   |-- branch.h
|   |-- echo.h
|   |-- errorhandlers.h
|   |-- filter.h
|   |-- flowy.h
|   |-- ftreader.h
|   |-- grouper.h
|   |-- groupfilter.h
|   |-- merger.h
|   |-- pipeline.h
|   |-- ungrouper.h
|   `-- utils.h
|-- scripts/
|   |-- queries/
|   |   |-- build-ftp-tcp-session.py*
|   |   |-- build-http-octets.py*
|   |   |-- build-http-tcp-session.py*
|   |   `-- build-https-tcp-session.py*
|   `-- generate-functions.py*
|-- src/
|   |-- base.c
|   |-- branch.c
|   |-- echo.c
|   |-- errorhandlers.c
|   |-- filter.c
|   |-- flowy.c
|   |-- ftreader.c
|   |-- grouper.c
|   |-- groupfilter.c
|   |-- merger.c
|   |-- ungrouper.c
|   `-- utils.c
|-- CMakeLists.txt
|-- Doxyfile
|-- Makefile
`-- README.md
\end{lstlisting}

CMake compilation creates a \lstinline!.build/! and puts the executable
binary in \lstinline!bin/!.\\The auto generated C sources and headers
goto \lstinline!.build/! as well and are automatically included and
linked in the final binary.

\begin{lstlisting}
# custom command to prepare auto-generated sources
add_custom_command (
  OUTPUT ${CMAKE_CURRENT_BINARY_DIR}/auto-assign.h
         ${CMAKE_CURRENT_BINARY_DIR}/auto-assign.c
         ${CMAKE_CURRENT_BINARY_DIR}/auto-comps.h
         ${CMAKE_CURRENT_BINARY_DIR}/auto-comps.c
  COMMAND python ${CMAKE_SOURCE_DIR}/scripts/generate-functions.py
  COMMENT "Generating: auto-comps{h,c} and auto-assign.{h,c}"
)
\end{lstlisting}

CMake compilation also runs the build query scripts defined in
\lstinline!scripts/queries/! to generate some examples \lstinline!JSON!
queries and moves them to the \lstinline!examples/! folder ready to be
used by the \lstinline!bin/engine! binary.

\begin{lstlisting}
# custom command to generate examples
file(GLOB pyFILES ${CMAKE_SOURCE_DIR}/scripts/queries/*.py)
foreach(pyFILE ${pyFILES})
  set(query "${pyFILE}_query")
  add_custom_command (
    OUTPUT ${query}
    WORKING_DIRECTORY ${CMAKE_SOURCE_DIR}/examples/
    COMMAND python ${pyFILE}
    COMMENT "Generating: JSON example query using ${pyFILE}"
  )
  list(APPEND queryFILES ${query})
endforeach(pyFILE)
\end{lstlisting}

In order to avoid doing -

\begin{lstlisting}
[engine] >> mkdir .build
[engine] >> cd .build
[.build] >> cmake ..
[.build] >> make
[.build] >> cd ..
[engine] >> rm -r .build
\end{lstlisting}

taking inspiration from Dirk Joel-Luchini Colbry {[}1{]}, I am using a
Makefile that calls CMake to automate this operation.

\begin{lstlisting}
make: build/Makefile
    (cd .build; make)

build/Makefile: build
    (cd .build; cmake -D CMAKE_PREFIX_PATH=$(CMAKE_PREFIX_PATH) ..)

build:
    mkdir -p .build
\end{lstlisting}

Additional targets to \lstinline!clean! and generate Doxygen
documentation \lstinline!doc! also exist.\\The generated documentation
goes in \lstinline!doc/! and is subsequently deleted by
\lstinline!make clean!.

\begin{lstlisting}
doc: Doxyfile
    (mkdir -p doc; doxygen Doxyfile)

clean:
    rm -f -r .build
    rm -f -r bin
    rm -f -r doc
    rm -f -r examples/*.json
\end{lstlisting}

Makefile can also take \lstinline!CMAKE_PREFIX_PATH! as an argument and
pass it on to CMake.\\\lstinline!CMAKE_PREFIX_PATH! is used to supply
arbitrary location of external libraries and include PATHS.

Resources:

{[}1{]}
\href{https://wiki.hpcc.msu.edu/display/~colbrydi@msu.edu/2010/08/19/Cmake+Makefile}{CMake
Makefile →}
