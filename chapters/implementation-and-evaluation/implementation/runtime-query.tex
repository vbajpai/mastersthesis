The complete \texttt{JSON} query is now read in at \emph{runtime}. The
branchsets and each \ac{DNF} expression of the stage is a \texttt{JSON array}
as shown in listing \ref{lst:fv2-fquery-json}. A \ac{DNF} expression is a
disjunction of conjunctive clauses. Each clause in the expression is OR'd
while each term within the clause is AND'd together. This terminology is more
intuitive to modules, submodules and rulesets as used previously by F$(v1)$.

\lstset{caption=F$(v2)$: Flow Query in \texttt{JSON},
				tabsize=2, numbers=left,stepnumber=1,
        basicstyle=\tiny\ttfamily, numberstyle=\ttfamily\color{gray},
        keywordstyle=\color{blue}, frame=shadowbox,
        rulesepcolor=\color{black}, label=lst:fv2-fquery-json,
        aboveskip=20pt, captionpos=b, upquote=true}
\begin{lstlisting}
{
  "branchset": [
    { "filter": {
        "dnf-expr": [
          { "clause": [
              { "term": { ... } },
              { "term": { ... } }
            ]
          }
        ]
      },
      ...
    },
  ],
  "merger": {
    "dnf-expr": [
      { "clause": [
          { "term": { ... } },
          { "term": { ... } }
        ]
      }
    ]
  },
  "ungrouper": {}
}
\end{lstlisting}

\texttt{json-c}\footnote{\url{http://oss.metaparadigm.com/json-c/}} is used
to parse such a query file read into memory by calling
\texttt{parse\_json\_query(\ldots)} \marginpar{parsing using json-c}. The
\texttt{json\_query} is then used to prepare the \texttt{struct flowquery}
used by the pipeline stages as shown in listing \ref{lst:fv2-json-c}.  The
\texttt{json\_query} struct is just an intermediate.

\lstset{caption=F$(v2)$: Parsing \texttt{JSON} query using \texttt{json-c},
				tabsize=2, language=C, numbers=left,stepnumber=1,
        basicstyle=\tiny\ttfamily, numberstyle=\ttfamily\color{gray},
        keywordstyle=\color{blue}, frame=shadowbox,
        rulesepcolor=\color{black}, label=lst:fv2-json-c,
        aboveskip=20pt, captionpos=b, upquote=true}
\begin{lstlisting}
struct json {
  size_t                                 num_branches;
  size_t                                 num_merger_clauses;

  struct json_branch**                   branchset;
  struct json_merger_clause**            merger_clauseset;
};

struct json_branch {
  size_t                                 num_filter_clauses;
  size_t                                 num_grouper_clauses;
  size_t                                 num_aggr_clause_terms;
  size_t                                 num_groupfilter_clauses;

  struct json_filter_clause**            filter_clauseset;
  struct json_grouper_clause**           grouper_clauseset;
  struct json_aggr_term**                aggr_clause_termset;
  struct json_groupfilter_clause**       groupfilter_clauseset;
};

struct json*
json_query = parse_json_query(param_data->query_mmap);

struct flowquery*
fquery = prepare_flowquery(param_data->trace, json_query);
\end{lstlisting}




The \texttt{JSON} query is verbose and cumbersome to write manually. The
python parser will eventually emit this intermediate format, so the next
logical \marginpar{generating json queries using python} step is to generate
the query from python. A python module (\texttt{scripts/queries/pipeline.py})
that encapsulates each pipeline stage as a separate class is shown in listing
\ref{lst:fv2-pipeline-module}.  Scripts that generate \texttt{JSON} queries
can \texttt{import} this module to reduce code redundancy.

\lstset{caption=F$(v2)$: Python Pipeline Module,
				tabsize=2, language=Python, numbers=left,stepnumber=1,
        basicstyle=\tiny\ttfamily, numberstyle=\ttfamily\color{gray},
        keywordstyle=\color{blue}, frame=shadowbox,
        rulesepcolor=\color{black}, label=lst:fv2-pipeline-module,
        aboveskip=20pt, captionpos=b, upquote=true}
\begin{lstlisting}
def protocol(name):
  return socket.getprotobyname(name)

class FilterRule: ...
class GrouperRule: ...
class AggregationRule: ...
class GroupFilterRule: ...
class MergerRule: ...
\end{lstlisting}

A sample script to generate a query is shown in listing
\ref{lst:fv2-python-scripts-json-queries}. Each \texttt{clause} is a list of
python objects of a specific class of the \texttt{pipeline} module. At this
point, \marginpar{sample scripts} the python parser just needs to create each
stage term objects and the script will take care to emit the \texttt{JSON}.
Example scripts to generate different queries are provided in
\texttt{scripts/queries/}.


\lstset{caption=F$(v2)$: Python Scripts to Generate \texttt{JSON} queries,
				tabsize=2, language=Python, numbers=left,stepnumber=1,
        basicstyle=\tiny\ttfamily, numberstyle=\ttfamily\color{gray},
        keywordstyle=\color{blue}, frame=shadowbox,
        rulesepcolor=\color{black}, label=lst:fv2-python-scripts-json-queries,
        aboveskip=20pt, captionpos=b, upquote=true}
\begin{lstlisting}
import json
from pipeline import FilterRule, GrouperRule, AggregationRule
from pipeline import GroupFilterRule, MergerRule
from pipeline import protocol

if __name__ == '__main__':

  # filter stage
  term1 = {'term': vars(FilterRule(...))}
  clause1 = {'clause': [term1] + ...}
  filter1 = {'dnf-expr': [clause1] + ...}

  # grouper stage
  term1 = {'term': vars(GrouperRule(...))}
  clause1 = {'clause': [term1] + ...}
  term1 = {'term': vars(AggregationRule(...))}
  aggregation1 = {'clause': [term1] + ...}
  grouper1 = {'dnf-expr': [clause1] + ..., 'aggregation': aggregation1}i

  # group filter stage
  term1 = {'term': vars(GroupFilterRule(...))}
  clause1 = {'clause': [term1] + ...}
  gfilter1 = {'dnf-expr': [clause1] + ...}

  branchset = []
  branchset.append(
                    {
                      'filter': filter1,
                      'grouper': grouper1,
                      'groupfilter': gfilter1,
                    }, ...
                  )

  # merger stage
  term1 = {'term': vars(MergerRule(...))}
  clause1 = {'clause': [term1] + ...}
  merger = {'dnf-expr': [clause1] + ...}

  # ungrouper stage
  ungrouper = {}

  query = {
            'branchset': branchset,
            'merger': merger,
            'ungrouper': ungrouper
          }
\end{lstlisting}






The mapping of the \texttt{JSON} query to the structs defined in the execution
engine is trickier than it looks. When reading the \texttt{JSON} query at
runtime, the field offsets of the NetFlow $v5$ record struct are read in as
strings from the \texttt{JSON} query. A utility function
\texttt{get\_offset(\ldots)} was thus \marginpar{runtime query internals}
introduced that maps the read names to struct offsets as shown in listing
\ref{lst:fv2-json-parsing-utils}.  In addition, the type of each offset and
the operations are also read from the \texttt{JSON} query as strings. This
information is saved and thus used by the engine using an \texttt{enum}
defined in \texttt{pipeline.h}.  Therefore, another utility function
\texttt{get\_enum(\ldots)} was defined to map this information to the unique
\texttt{enum} members as shown in listing \ref{lst:fv2-json-parsing-utils}.


\lstset{caption=F$(v2)$: \texttt{JSON} Parsing Utilities,
				tabsize=2, language=C, numbers=left,stepnumber=1,
        basicstyle=\tiny\ttfamily, numberstyle=\ttfamily\color{gray},
        keywordstyle=\color{blue}, frame=shadowbox,
        rulesepcolor=\color{black}, label=lst:fv2-json-parsing-utils,
        aboveskip=20pt, captionpos=b, upquote=true}
\begin{lstlisting}
size_t
get_offset(
           const char * const name,
           const struct fts3rec_offsets* const offsets
          ) {

  #define CASEOFF(memb)                       \
  if (strcmp(name, #memb) == 0)               \
    return offsets->memb

    CASEOFF(unix_secs);
    CASEOFF(unix_nsecs);
    ...

  return -1;
}

uint64_t
get_enum(const char * const name) {

  #define CASEENUM(memb)                      \
  if (strcmp(name, #memb) == 0)               \
    return memb

  CASEENUM(RULE_S1_8);
  CASEENUM(RULE_S1_16);
  ...
  CASEENUM(RULE_S2_8);
  CASEENUM(RULE_S2_16);
  ...
  CASEENUM(RULE_ABS);
  CASEENUM(RULE_REL);
  CASEENUM(RULE_NO);
  ...
  CASEENUM(RULE_EQ);
  CASEENUM(RULE_NE);
  ...
  CASEENUM(RULE_STATIC);
  CASEENUM(RULE_COUNT);
  ...
  CASEENUM(RULE_ALLEN_BF);
  CASEENUM(RULE_ALLEN_AF);
  ...
  return -1;
}
\end{lstlisting}

The \texttt{JSON} query can also trigger and disable the stages at runtime.
This means that one only has to supply the constructs that one wishes to use.
The constructs that are not defined in the \texttt{JSON} query are inferred by
the engine as a \marginpar{disabling pipeline stages at runtime} disable
request. The execution engine uses disable flags that are turned on when the
\texttt{JSON} query is parsed as shown in listing
\ref{lst:fv2-disable-runtime}. These flags are used throughout the engine to
only enable the requested functionality. It is important to note though that
the engine is currently not smart to understand the interdependency amongst
the stages. For instance, disabling the merger when the ungrouper is kept
enabled will lead to undefined behavior.

\lstset{caption=F$(v2)$: Disabling Pipeline Stages at Runtime,
				tabsize=2, language=C, numbers=left,stepnumber=1,
        basicstyle=\tiny\ttfamily, numberstyle=\ttfamily\color{gray},
        keywordstyle=\color{blue}, frame=shadowbox,
        rulesepcolor=\color{black}, label=lst:fv2-disable-runtime,
        aboveskip=20pt, captionpos=b, upquote=true}
\begin{lstlisting}
struct json*
parse_json_query(const char* const query_mmap) {

  ...
  /* filter */
  struct json_object* filter = json_object_object_get(branch_json, "filter");
  if (filter != NULL) filter_enabled = true;
  if (filter_enabled) ...

  /* grouper */
  struct json_object* grouper = json_object_object_get(branch_json, "grouper");
  if (grouper != NULL) grouper_enabled = true;
  if (grouper_enabled) ...

  /* grouper aggregation */
  if (grouper_enabled) {
    struct json_object* aggr = json_object_object_get(grouper, "aggregation");
    if (aggr != NULL) groupaggregations_enabled = true;
    if (groupaggregations_enabled) ...
  }

  /* group filter */
  struct json_object* gfilter = json_object_object_get(branch_json, "groupfilter");
  if (gfilter != NULL) groupfilter_enabled = true;
  if (groupfilter_enabled) ...

  /* merger */
  struct json_object* merger = json_object_object_get(query, "merger");
  if (merger != NULL) merger_enabled = true;
  if (merger_enabled) ...

  /* ungrouper */
  struct json_object* ungrouper = json_object_object_get(query, "ungrouper");
  if (ungrouper != NULL) ungrouper_enabled = true;
  ...
}
\end{lstlisting}
