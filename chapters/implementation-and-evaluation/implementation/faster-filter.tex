\begin{figure}[h!]
  \begin{center}
    \includegraphics* [width=0.8\linewidth]{figures/benchmarks/filter-fv1-fv2}
    \caption{Filter Stage: F$(v1)$ vs F$(v2)$}
    \label{fig:fv1-fv2-filter}
  \end{center}
\end{figure}

The execution engine in F$(v1)$ used to read all the flow records of a
supplied trace into memory before starting the processing pipeline.  Since,
the filter stage uses the  supplied set of absolute rules to make a decision
on whether or not to keep a flow record; it had to pass through the whole
in-memory recordset \emph{again} to fill in the filter results. This technique
involves multiple linear runs on the trace and therefore slows down when the
ratio of number of filtered records to the total number of flow-records is
high as shown in figure \ref{fig:fv1-fv2-filter}.

The filter stage in F$(v2)$, therefore has now been merged with in-memory read
of the trace. This means, a decision on whether or not to make room for a
record in memory and eventually hold a pointer for it in filter results is
done upfront as soon as the record is read from the trace. In addition, if a
request to write the filter stage results to a flow-tools file has been made,
the writes are also made as soon the filter stage decision is available,
thereby  \marginpar{faster filter}  allowing reading-filtering-writing to
happen in $O(n)$ time, where $n$ is the number of records in the trace. It is
important to note that the filtered records are saved in common location from
where they are referenced by each branch. This helps keep the memory costs at
a minimum when multiple branches are involved. A publically available Netflow
$v5$ trace
\footnote{\url{http://traces.simpleweb.org/traces/netflow/netflow1/netflow000.tar.bz2}}
was used to compare the performance of the new filter in F$(v2)$ with that
from F$(v1)$ as shown in figure \ref{fig:fv1-fv2-filter}. The trace has around
$20M$ flow-records. The queries executed are available in source repository
\footnote{\url{http://github.com/vbajpai/mthesis-src/}} and have been omitted
from here for brevity reasons. 
