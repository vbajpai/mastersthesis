\begin{figure}[h!]
  \begin{center}
    \includegraphics*
    [width=0.8\linewidth]{figures/benchmarks/grouper-fv1-fv2-bsearch}
    \caption{Grouper Stage: F$(v1)$ vs F$(v2)$}
    \label{fig:fv1-fv2-grouper-bsearch}
  \end{center}
\end{figure}

The \texttt{qsort} and \texttt{bsearch} approach as suggested in F$(v1)$,
applies the sorting and searching only on the first grouping rule of the flow
query. This results in a linear pass from the index retrieved by the
\texttt{bsearch} till either the end of the filtered recordset or until the
first rule fails. The linear pass is processed for each potential group. Such
an approach works well when \marginpar{faster grouping in a generic case} the
ratio of the number of filtered records to the number of groups is low.
However for higher ratios, the time increases exponentially as can be seen in
figure \ref{fig:fv1-fv2-grouper-bsearch}. A better approach, as implemented in
F$(v2)$ is to sort the filtered recordset on all the requested grouping rules.
A comparator that performs nested \texttt{qsort\_r} comparisons to sort on
each grouping rule is shown in listing \ref{lst:fv2-qsortr-nested}. This helps
the execution engine perform a nested \texttt{bsearch} to reduce the linear
pass to a fairly small filtered recordset. However, the number of elements
upon which the search has to be performed needs to be known at each level of
the binary lookup. In order to avoid a linear run to count the number of
elements, the parent level \texttt{bsearch} invocation returns a boundary with
the first and last element to enable such a calculation. The algorithm to find
the first and last element using \texttt{bsearch} is shown in listing
\ref{lst:bsearchfirst} and \ref{lst:bsearchlast} respectively.


\lstset{caption=F$(v2)$: Nested qsort\_r Comparator,
				tabsize=2, language=C, numbers=left,stepnumber=1,
        basicstyle=\tiny\ttfamily, numberstyle=\ttfamily\color{gray},
        keywordstyle=\color{blue}, frame=shadowbox,
        rulesepcolor=\color{black}, label=lst:fv2-qsortr-nested,
        aboveskip=20pt, captionpos=b}
\begin{lstlisting}
#if defined(__APPLE__) || defined(__FreeBSD__)
  int qsort_comp(void *thunk, const void *e1, const void *e2) {
#elif defined(__linux)
  int qsort_comp(const void *e1, const void *e2, void *thunk) {
#endif
    ...
    for (int i = 0; i < clause->num_terms; i++) {
        ...
    #if defined(__APPLE__) || defined(__FreeBSD__)
        result = gtype->qsort_comp((void*)term->field_offset2, e1, e2);
    #elif defined(__linux)
        result = gtype->qsort_comp(e1, e2, (void*)term->field_offset2);
    #endif
        if (result != 0) break;
    }
    return result;
  }
\end{lstlisting}

The same trace as used in section \ref{sec:faster-filter} is used to compare
the performance of the F$(v1)$ and F$v(2)$ grouper stage. The queries are are
available in the source code repository.  The evaluation is shown in figure
\ref{fig:fv1-fv2-grouper-bsearch}. It is clear that when the ratio of the
number of groups formed to the input filtered records is low, the performance
is comparable. However on higher ratios, the times taken are far apart. It is
because, the frequency of linear passes made in F$(v1)$ increases with the
number of potential groups. However with the nested \texttt{bsearch} in
F$(v2)$ the linear pass is small enough to not make a huge difference.

\begin{center}
\begin{minipage}{.44\textwidth}
  \lstset{caption=\texttt{bsearch\_first\_item},
					tabsize=2, language=C, numbers=left,stepnumber=1,
					basicstyle=\tiny\ttfamily,numberstyle=\ttfamily\color{gray},
					keywordstyle=\color{blue},frame=shadowbox, rulesepcolor=\color{black},
				  label=lst:bsearchfirst, aboveskip=20pt, captionpos=b}
	\begin{lstlisting}
mid = (size_t)floor((low+high)/2.0);

if (comparison > 0)
  low = mid + 1;

else if (comparison < 0)
  high = mid - 1;

else if (low != mid)
  high = mid;

else return mid;
	\end{lstlisting}
\end{minipage}
\hfill
\begin{minipage}{.44\textwidth}
  \lstset{caption=\texttt{bsearch\_last\_item},
					tabsize=2, language=C, numbers=left,stepnumber=1,
					basicstyle=\tiny\ttfamily, numberstyle=\ttfamily\color{gray},
					keywordstyle=\color{blue},frame=shadowbox, rulesepcolor=\color{black},
				  label=lst:bsearchlast, aboveskip=20pt, captionpos=b}
	\begin{lstlisting}
mid = (size_t)ceil((low+high)/2.0);

if (comparison > 0)
  low = mid + 1;

else if (comparison < 0)
  high = mid - 1;

else if (high != mid)
  low = mid;

else return mid;
	\end{lstlisting}
\end{minipage}
\end{center}

The grouping approach has further been optimized when the filtered records are
grouped for equality. In such a scenario, the need to search for unique
records and a  subsequent binary search goes away.  The groups can now be
formed in $O(n)$ time with a more involved preprocessing step taking
$O(p*n*lg(n)$ where is $n$ is the number of \marginpar{faster grouping in a
specific case} filtered records, and $p$ is the number of grouping rules. The
performance evaluation of the grouper handling this special case against its
behavior when handling generic cases is shown in figure
\ref{fig:fv1-fv2-grouper}. Profiling the performance of the grouper on queries
that produce higher ratios reveal that a significant amount of time is taken
in \texttt{bsearch}. The special case of equality comparisons eliminate these
calls and further reduce the time in the long tail as shown in figure
\ref{fig:fv1-fv2-grouper}.

\begin{figure}[h!]
  \begin{center}
    \includegraphics*
    [width=0.8\linewidth]{figures/benchmarks/grouper-fv2-bsearch-fv2}
    \caption{Grouper Stage: F$(v2)$ (Generic) vs F$(v2)$ (Specific)}
    \label{fig:fv1-fv2-grouper}
  \end{center}
\end{figure}

