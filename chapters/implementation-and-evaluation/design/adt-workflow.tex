\begin{figure}[h!]
  \begin{center}
    \includegraphics* [width=1.0\linewidth]{figures/engine-workflow}
    \caption{F$(v2)$: Execution Engine Workflow}
    \label{fig:engine-workflow}
  \end{center}
\end{figure}

In order to keep the codebase maintainable, it was essential to design the
execution engine workflow in such a way so as to naturally map it to the
original pipeline model specification \cite{vmarinov:2009} as shown in figure
\ref{fig:engine-workflow}. Each stage of the pipeline is a separate
independent module blackboxed into one public interface function. Each stage
is also wrapped around conditional compilation macros to allow them to be
easily enabled/disabled during development if desired so.

\lstset{caption=F$(v2)$: Flow Query Composition using \ac{DNF} Expressions,
				tabsize=2, language=bash, numbers=left,stepnumber=1,
        basicstyle=\tiny\ttfamily, numberstyle=\ttfamily\color{gray},
        keywordstyle=\color{blue}, frame=shadowbox,
        rulesepcolor=\color{black}, label=lst:flowquery-dnf,
        aboveskip=20pt, captionpos=b, upquote=true}
\begin{lstlisting}
{
  "branchset": [
    { "filter": {
        "dnf-expr": [
          { "clause": [
              { "term": { ... } },
              { "term": { ... } }
            ]
          }
        ]
      },
      "grouper": {
        "dnf-expr": [
          { "clause": [
              { "term": { ... } },
              { "term": { ... } }
            ]
          }
        ],
        "aggregation": {
          "clause": [
            { "term": { ... } },
            { "term": { ... } }
          ]
        }
      },
      "groupfilter": {
        "dnf-expr": [
          { "clause": [
              { "term": { ... } },
              { "term": { ... } }
            ]
          }
        ]
      }
    },
  ],
  "merger": {
    "dnf-expr": [
      { "clause": [
          { "term": { ... } },
          { "term": { ... } }
        ]
      }
    ]
  },
  "ungrouper": {}
}
\end{lstlisting}

A \ac{DNF} is a disjunction of conjunctive clauses. The elements of the
conjuctive clauses are terms. Each stage of the pipeline is represented in the
flowquery as a \ac{DNF} expression as shown in listing
\ref{lst:flowquery-dnf}. The clauses \marginpar{flowquery composition using
dnf expressions} in the \ac{DNF} are OR'd together, while the terms in each
clauses themselves are AND'd. This terminology using clauses and terms is
more useful and intuitive over modules, submodules and rulesets used
previously by \ac{NFQL} and its implementations.

\lstset{caption=F$(v2)$: Flow Query Struct,
				tabsize=2, language=C, numbers=left,stepnumber=1,
        basicstyle=\tiny\ttfamily, numberstyle=\ttfamily\color{gray},
        keywordstyle=\color{blue}, frame=shadowbox,
        rulesepcolor=\color{black}, label=lst:fv2-flowquery-struct,
        aboveskip=20pt, captionpos=b}
\begin{lstlisting}
struct flowquery {
  size_t                                      num_branches;
  size_t                                      num_merger_clauses;

  struct branch**                             branchset;
  struct merger_clause**                      merger_clauseset;
  struct merger_result*                       merger_result;
  struct ungrouper_result*                    ungrouper_result;
};
\end{lstlisting}

The abstract objects that store the JSON query and the results that incubate
from each stage are designed to be self-descriptive and hierarchically
chainable. The complete JSON query information for instance, is held within
the \texttt{flowquery} \marginpar{flowquery and branch struct} struct as shown
in listing \ref{lst:fv2-flowquery-struct}. Each individual branch of the
flowquery itself is described in a \texttt{branch} struct. A collection of
these \texttt{branch} structs are referenced in the parent \texttt{flowquery}
struct. All the clauses of the \ac{DNF} expression are clubbed into
\texttt{X\_clauseset}, where \texttt{X} can be any stage as shown in listing
\ref{lst:fv2-branch-struct}.

\lstset{caption=F$(v2)$: Branch Struct,
				tabsize=2, language=C, numbers=left,stepnumber=1,
        basicstyle=\tiny\ttfamily, numberstyle=\ttfamily\color{gray},
        keywordstyle=\color{blue}, frame=shadowbox,
        rulesepcolor=\color{black}, label=lst:fv2-branch-struct,
        aboveskip=20pt, captionpos=b}
\begin{lstlisting}
struct branch {

/* -----------------------------------------------------------------------*/
/*                              inputs                                    */
/* -----------------------------------------------------------------------*/

  int                                         branch_id;
  struct ftio*                                ftio_out;
  struct ft_data*                             data;

  size_t                                      num_filter_clauses;
  size_t                                      num_grouper_clauses;
  size_t                                      num_aggr_clause_terms;
  size_t                                      num_groupfilter_clauses;

  struct filter_clause**                      filter_clauseset;
  struct grouper_clause**                     grouper_clauseset;
  struct aggr_term**                          aggr_clause_termset;
  struct groupfilter_clause**                 groupfilter_clauseset;

/* -----------------------------------------------------------------------*/




/* -----------------------------------------------------------------------*/
/*                               output                                   */
/* -----------------------------------------------------------------------*/

  struct filter_result*                       filter_result;
  struct grouper_result*                      grouper_result;
  struct groupfilter_result*                  gfilter_result;

/* -----------------------------------------------------------------------*/
};
\end{lstlisting}

A call to the public interface function of each stage returns a
\texttt{X\_result} struct object as shown in listing \ref{lst:fv2-interfaces}.
The \texttt{X\_result} objects encapsulate all elements \marginpar{public
interfaces} of the stage into one single entity as shown in listing
\ref{lst:fv2-interfaces} to easily allow them to be passed around and for easy
maintainbility of in-memory object stores.

\lstset{caption=F$(v2)$: Public Interfaces,
				tabsize=2, language=C, numbers=left,stepnumber=1,
        basicstyle=\tiny\ttfamily, numberstyle=\ttfamily\color{gray},
        keywordstyle=\color{blue}, frame=shadowbox,
        rulesepcolor=\color{black}, label=lst:fv2-interfaces,
        aboveskip=20pt, captionpos=b}
\begin{lstlisting}
struct grouper_result*
grouper(...) {...}

struct groupfilter_result*
groupfilter(...) {...}

struct merger_result*
merger(...) {...}

struct ungrouper_result*
ungrouper(...) {...}
\end{lstlisting}

Each result struct holds information about the number of flow records that
passed the stage and pointers to each such flow records. Since the group
filter and merger stages do not work on the individual flows but on a
collection; they take \marginpar{result structs} the \texttt{group} struct
that encapsulates a collection of similar flows as input arguments. It is
important to realize that the flow records themselves are never carried
forward from each stage to its subsequents, but only offset pointers to the
original flow trace are.

\lstset{caption=F$(v2)$: Result Structs,
				tabsize=2, language=C, numbers=left,stepnumber=1,
        basicstyle=\tiny\ttfamily, numberstyle=\ttfamily\color{gray},
        keywordstyle=\color{blue}, frame=shadowbox,
        rulesepcolor=\color{black}, label=lst:fv2-result-structs,
        aboveskip=20pt, captionpos=b}
\begin{lstlisting}
struct filter_result {
  uint32_t                                    num_filtered_records;
  char**                                      filtered_recordset;
};

struct grouper_result {
  char**                                      sorted_recordset;
  uint32_t                                    num_groups;
  struct group**                              groupset;
};

struct groupfilter_result {
  uint32_t                                    num_filtered_groups;
  struct group**                              filtered_groupset;
};

struct merger_result {
  uint32_t                                    num_group_tuples;
  size_t                                      total_num_group_tuples;
  struct group***                             group_tuples;
};

struct ungrouper_result {
  size_t                                      num_streams;
  struct stream**                             streamset;
};
\end{lstlisting}


The query fragment structs (\texttt{X\_clauseset}) used to get the result is
greedily deallocated soon after the stage returns to keep the in-memory usage
to the minimum. The \texttt{filter\_clauseset} \marginpar{greedy clauseset
deallocation} although is kept until the end of the grouper stage since it
helps the grouper aggregation stage make decisions on whether a linear pass
through the flow trace is required to aggregate a column that may have been
already a criteron for the filter stage.

\lstset{caption=F$(v2)$: Greedy Ruleset Deallocation,
				tabsize=2, language=C, numbers=left,stepnumber=1,
        basicstyle=\tiny\ttfamily, numberstyle=\ttfamily\color{gray},
        keywordstyle=\color{blue}, frame=shadowbox,
        rulesepcolor=\color{black}, label=lst:fv2-greedy-dealloc,
        aboveskip=20pt, captionpos=b}
\begin{lstlisting}
branch->grouper_result = grouper(...);
if (branch->grouper_result == NULL)  ...
else {
  /* free filter clauses */
  /* free grouper clauses */
  /* free grouper aggregation clause */
}

branch->gfilter_result = groupfilter(...);
if (branch->gfilter_result == NULL) ...
else {
  /* free group filter clauses */
}

fquery->merger_result = merger(...);
if (fquery->merger_result == NULL) ...
else {
    /* free merger clauses */
}
\end{lstlisting}
