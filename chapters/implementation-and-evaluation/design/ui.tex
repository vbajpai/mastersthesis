It is essential to allow the interface to be intuitive to any new user who is
interested in using the tool for network analysis. In essence, this is
achieved \marginpar{pretty usage help, tracking invalid options} using the
standard \texttt{getopt\_long} call to allow both short and long option
arguments. The execution engine appropriately displays the usage help when
insufficient number of arguments are provided as shown in listing
\ref{lst:fv2-ui}.  The engine is also interactive to help one choose the right
switches with required options.

\lstset{caption=F$(v2)$: User Interface,
				tabsize=2, numbers=left,stepnumber=1,
        basicstyle=\tiny\ttfamily, numberstyle=\ttfamily\color{gray},
        keywordstyle=\color{blue}, frame=shadowbox,
        rulesepcolor=\color{black}, label=lst:fv2-ui,
        aboveskip=20pt, captionpos=b}
\begin{lstlisting}
$ bin/engine
usage: bin/engine [OPTIONS] queryfile tracefile      query the specified trace
   or: bin/engine [OPTIONS] queryfile -              read the trace from stdin

OPTIONS:
-z, --zlevel             change the compression level (default:5)
-d, --dirpath            save the results as flow-tools files in given dirpath
-D, --debug              enable debugging mode
-v, --verbose            increase the verbosity level
-V, --version            output version information and exit
-h, --help               display this help and exit

$ bin/engine queryfile tracefile --foo
bin/engine: invalid option --foo

$ bin/engine queryfile tracefile --verbose
bin/engine: option --verbose requires an argument

$ bin/engine queryfile tracefile --verbose=5
ERROR: valid verbosity levels: (1-3)
\end{lstlisting}

Since the execution engine largely depends on the sanity of the query and
trace files passed to it as arguments, it is essential to let the input files
pass \marginpar{consistency checks} through a level of consistency check
before going forward with the processing pipeline to avoid any undefined
behavior as shown in listing \ref{lst:fv2-consistency-checks}.

\lstset{caption=F$(v2)$: Consistency Checks,
				tabsize=2, numbers=left,stepnumber=1,
        basicstyle=\tiny\ttfamily, numberstyle=\ttfamily\color{gray},
        keywordstyle=\color{blue}, frame=shadowbox,
        rulesepcolor=\color{black}, label=lst:fv2-consistency-checks,
        aboveskip=20pt, captionpos=b}
\begin{lstlisting}
$ bin/engine README.md tracefile
ERROR: json_tokener_parse_ex(...)

$ bin/engine queryfile README.md
ERROR: ftio_init(...)
\end{lstlisting}

With a software undergoing such a rapid pace of development, it's helpful to
be \marginpar{backtrace on graceful exits} able to see  the inner workings of
each stage of the pipeline during a debugging lifecycle. As such, the engine
echoes a backtrace whenever it fails gracefully as shown in listing
\ref{lst:fv2-backtrace}.

\lstset{caption=F$(v2)$: Backtraces,
				tabsize=2, numbers=left,stepnumber=1,
        basicstyle=\tiny\ttfamily, numberstyle=\ttfamily\color{gray},
        keywordstyle=\color{blue}, frame=shadowbox,
        rulesepcolor=\color{black}, label=lst:fv2-backtrace,
        aboveskip=20pt, captionpos=b}
\begin{lstlisting}
$ bin/engine foo bar
ERROR: open(...)
BACKTRACE:
0   engine             0x000000010bc891c2 print_trace + 34
1   engine             0x000000010bc893cb errExit + 395
2   engine             0x000000010bc898de read_param_data + 174
3   engine             0x000000010bc8c600 main + 80
4   engine             0x000000010bc6e054 start + 52
\end{lstlisting}

\begin{figure}[h!]
  \begin{center}
    \includegraphics* [width=0.6\linewidth]{figures/engine-verbosity}
    \caption{F$(v2)$: Verbosity Levels Workflow}
    \label{fig:engine-verbosity}
  \end{center}
\end{figure}

The engine also allows to increase the amount of output generated using a
number of verbosity levels. A specific function is designed to handle the
output generation of each stage of the pipeline as shown in figure
\ref{fig:engine-verbosity}.  In its default state, the engine only outputs the
resultant streams \footnote{A \texttt{stream} is a collection of flow-records
of a group unfolded by the ungrouper}\footnote{A \texttt{streamset} is
collection of all the streams unfolded by the ungrouper} \marginpar{debug and
verbosity levels} of flow records. A debug (or \texttt{--verbose=3}) level
engine execution is shown in listing \ref{lst:fv2-debug}. In addition to
echoing the flow (or group) records resulting from each stage, it also echoes
the results of each intermediate stage alongwith the original trace that was
passed to it. With \texttt{--verbose=2}, the echo of the original trace is
pruned, while intermediate results get pruned with \texttt{--verbose=1}.

\lstset{caption=F$(v2)$: Debugging, tabsize=2, numbers=left,stepnumber=1,
  basicstyle=\tiny\ttfamily, numberstyle=\ttfamily\color{gray},
  keywordstyle=\color{blue}, frame=shadowbox, rulesepcolor=\color{black},
  label=lst:fv2-debug, aboveskip=20pt, captionpos=b}

\begin{lstlisting}
$ bin/engine tracefile queryfile --debug

# capture hostname:     ihp.jacobs.jacobs-university.de ...

No. of Filtered Records: ...
No. of Sorted Records: ...
No. of Unique Records: ...
No. of Groups: (Verbose Output): ...

... 0     216.137.61.203  80    0     192.168.0.135 ...
... 0     216.137.61.203  80    0     192.168.0.135 ...

... 0     8.12.214.126    80    0     192.168.0.135 ...
... 0     8.12.214.126    80    0     192.168.0.135 ...

No. of Groups: 32 (Aggregations): ...

... 0     216.137.61.203  80    0     192.168.0.135 ...
... 0     8.12.214.126    80    0     192.168.0.135 ...

No. of Filtered Groups: (Aggregations): ...
No. of (to be) Matched Groups: ...

... 0     192.168.0.135   0     0     204.160.123.126 80    ...
... 0     87.238.86.121   80    0     192.168.0.135   0     ...

No. of Merged Groups: 3 (Tuples): ...

... 0     192.168.0.135   0     0     216.46.94.66    80    ...
... 0     216.46.94.66    80    0     192.168.0.135   0     ...

No. of Streams: ...
\end{lstlisting}

The flow-records echoed to the standard output can also be written to a
Netflow $v5$ flow-tools file. The \texttt{--dirpath} switch allows one to
provide a directory path where the results can be stored.  Each stream is
stored as \marginpar{writing results as flow-tools files} its own file with an ID to
disambiguate it. Results from each stage of the pipeline can also be written
to separate files with the increase in the verbosity level. In fact,
\texttt{--dirpath} and \texttt{--verbose} work well together to adjust the
level at which the writes are to be made. Such a feature allows one to
blackbox each stage's functionality and analyze it independently, which can be
useful not only for unit testing but also for performance evaluations
purposes.


\begin{figure}[h!]
  \begin{center}
    \includegraphics* [width=0.8\linewidth]{figures/benchmarks/zlevel}
    \caption{F$(v2)$: \texttt{z-level} Effect on Performance}
    \label{fig:engine-zlevel}
  \end{center}
\end{figure}


The engine uses the \texttt{zlib} \cite{rfc1950} software library to compress
the results written to the flow-tools files. \texttt{zlib} supports $9$
compression levels with $9$ being the highest. The \ac{NFQL} engine supports
\texttt{--zlevel} \marginpar{adaptable compression levels} switch to allow the
user to supply its desired choice of the of the compression level. A default
level of $5$ is used for writes if the switch is not supplied during runtime.
Figure \ref{fig:engine-zlevel} shows the time taken to write a sample of
records passing the filter stage for each \texttt{z-level}. It goes to show
that each level adds its own performance overhead and must be used with
discretion.
